%resumo.tex
\cleardoublepage
\addcontentsline{toc}{chapter}{Resumo}
\chapter*{Resumo}
\thispagestyle{fancy}
Os motores de busca de propósito geral, como o Google, estão presentes no cotidiano de várias pessoas e são uma maneira fácil de buscar por conteúdo, solução de problemas ou até mesmo para fins de tradução. No entanto, estes motores de busca apresentam a limitação de não fornecer meios de recuperação de informações específicas e em formato estruturado. Para estes casos, há os motores de busca de propósito específico, que são projetados especialmente para atender este tipo de necessidade. No entanto, tais projetos podem ser custosos, seja por escassez de mão de obra qualificada ou pela natureza desafiadora do projeto. Este trabalho visa facilitar a criação e manutenção de sistemas de recuperação estruturada na Web através do uso de uma linguagem específica de domínio baseada em XML, utilizada para geração de código de \emph{web scrapers} em Python. Os resultados mostram o ganho de produtividade que um programador, mensurado em linhas de código escritas, tem ao utilizar a ferramenta desenvolvida neste trabalho pode chegar à mais de 30\%. Outros resultados qualitativos discutem o ganho de produtividade ao utilizar uma linguagem específica de domínio, em que o programador não precisa preocupar-se com todos os detalhes inerentes à uma linguagem de propósito geral.
	
	\textbf{Palavras-chave:} scraper, web, python, xml, wpt, dsl, recuperação de informação

\cleardoublepage
\addcontentsline{toc}{chapter}{Abstract}
\chapter*{Abstract}
\thispagestyle{fancy}
General purpose search engines, like Google, are present in daily people's life and they are a easy way to search content, problem solving and even for translation needs. However, this kind of search engine has a limitation in not offering ways to retrieve specific information in a structured format. For these cases, there are specific purpose search engines, which are designed specially to attend this kind of need. But these projects can be expensive because of specialized programmer lacking or by the project's challenging nature. This work aims to facilitate the structured web information retrieval sytems creation and maintenance by using a XML domain specific language for web scraper code generation in Python. The quantitative results shows an 30\% productive gain, in terms of number of lines coded when the programmers use the tool developed in this work. Another qualitative results discuss about the productivity gain when the programmer use a domain specific language and do not need to worry about all details present in a general purpose language.

	\textbf{Keywords:} scraper, web, python, xml, wpt, dsl, information retrieval
