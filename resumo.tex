%resumo.tex
\cleardoublepage
\addcontentsline{toc}{chapter}{Resumo}
\chapter*{Resumo}
\thispagestyle{fancy}
Os motores de busca de propósito geral, como o Google, estão presentes no cotidiano de várias pessoas e são uma maneira fácil de buscar por conteúdo, solução de problemas ou até mesmo para fins de tradução. No entanto, esses motores de busca apresentam a limitação de não fornecer meios de recuperação de informações específicas e em formato estruturado. Para esses casos, existem os motores de busca de propósito específico, que são projetados especialmente para atender esse tipo de necessidade. No entanto, tais projetos podem ser custosos, seja por escassez de mão de obra qualificada ou pela natureza desafiadora do projeto. Este trabalho tem como objetivo principal facilitar a criação e manutenção de sistemas de recuperação estruturada na Web através do uso de uma linguagem específica de domínio baseada em XML, utilizada para geração de código de \emph{web scrapers} em Python. A geração de código na linguagem Python a partir do código da linguagem específica de domínio baseada em XML é feita por uma ferramenta desenvolvida neste trabalho, que se integra ao Scrapy, um framework para recuperação de informações na Web. Os resultados mostram o ganho de produtividade que um programador, mensurado em linhas de código escritas, tem ao utilizar a ferramenta desenvolvida neste trabalho pode chegar à mais de 30\%. Outros resultados qualitativos discutem o ganho de produtividade ao utilizar uma linguagem específica de domínio, em que o programador não precisa preocupar-se com todos os detalhes inerentes à uma linguagem de propósito geral e com isso conseguem alcançar uma maior produtividade no desenvolvimento de soluções para recuperação de informação estruturada na Web.
	
	\textbf{Palavras-chave:} scraper, web, python, xml, wpt, dsl, recuperação de informação

\cleardoublepage
\addcontentsline{toc}{chapter}{Abstract}
\chapter*{Abstract}
\thispagestyle{fancy}
General purpose search engines, like Google, are present in daily people's life and they are an easy way to search content, problem solving and even for translation needs. However, this kind of search engine has a limitation in not offering ways to retrieve specific information in a structured format. For these cases, there are specific purpose search engines, which are designed specially to attend this kind of need. But these projects can be expensive because of specialized programmer lacking or by the project's challenging nature. This work aims to facilitate the structured web information retrieval sytems creation and maintenance by using a XML domain specific language for web scraper code generation in Python. The Python code generation from an XML-based domain specific language is done by a tool developed with this work which has an integration with Scrapy, an web-based information retrieval framework. The quantitative results shows an 30\% productive gain, in terms of number of lines coded when the programmers use the tool developed in this work. Another qualitative results discuss about the productivity gain when the programmer use a domain specific language and do not need to worry about all details present in a general purpose language and then achieve an higher productivity level at development of estrutuctured web information retrieval solutions.

	\textbf{Keywords:} scraper, web, python, xml, wpt, dsl, information retrieval
