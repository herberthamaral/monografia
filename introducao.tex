% introdução.tex
\chapter{Introdução}
\thispagestyle{fancy}

A \gls{www} é um grande repositório de documentos contendo informações sobre as mais diversas áreas do conhecimento. Com a advento de motores de busca de propósito geral, como o Google, buscar informações na Web se tornou uma tarefa trivial e cotidiana.

Porém, mesmo com seus grandes poderes de busca, os motores de busca de propósito geral possuem uma capacidade limitada de recuperação de informações específicas. Em uma busca por "Altura do Michael Jordan em centímetros" ou "Lista no formato XML dos candidatos à deputado federal em 1998", o Google, por exemplo, retornará uma lista de referências de várias páginas na Web que possam vir a conter a informação, sem apresentar a informação no formato que o usuário deseja (como por exemplo, em formato XML).

Os motores de busca de propósito específico são desenvolvidos para sanar a limitação de obtenção de dados específicos e de forma estruturada que os motores de busca de propósito geral possuem.

As informações obtidas por motores de busca de propósito específico na Web podem ser úteis para várias organizações. Por exemplo, um determinado sistema de informações de uma empresa de transportes precisa armazenar as infrações cometidas pelos seus motoristas. No entanto, estas informações só estão disponíveis mediante consulta no sítio da Web da autoridade responsável pelo trânsito. Um motor de busca de propósito específico pode ser criado para realizar essas consultas e alimentar o banco de dados do sistema, sem necessidade de intervenção manual no sistema.

% ------------------------------

A recuperação de Informações na Web é uma tarefa fácil, porém trabalhosa e propensa a erros se feita manualmente por humanos. Fácil, pois humanos conseguem detectar padrões de informações, categorizá-las e armazená-las como quiserem. Trabalhosa, pois podem haver inúmeras páginas com grandes quantidades de informação, o que torna o processo de recuperação manual de informações caro, trabalhoso, tedioso e pouco confiável em alguns casos. Dependendo do número de motoristas e da frequência de atualizações das bases de dados locais, a empresa citada no exemplo anterior precisaria contratar algumas pessoas a mais somente para realizar esse tipo de tarefa.

Soluções computacionais como \glspl{webcrawler} ou \glspl{webscraper} fazem parte dos motores de busca de propósito específico  e ajudam na automação do processo de recuperação de informações estruturadas. No entanto, por serem soluções computacionais, demandam que sejam programadas e, geralmente, para um fim específico. Um \gls{webscraper} que foi originalmente desenvolvido para extrair informações de um determinado tipo de página não conseguirá extrair as mesmas informações se a estrutura e/ou organização da página mudarem.

Na prática, um \gls{webscraper} projetado para um sítio precisará sofrer modificações, de forma que consiga extrair informações caso a estrutura da página em \gls{html} mude, seja por uma mudança na estrutura de páginas do sítio ou pelo uso de um novo sítio que possa prover as mesmas informações.

\hyphenation{pro-ble-ma}
Com a recuperação de informações de forma manual, há o problema da falta de automação de processo, o que pode levar a um alto custo  de recuperação de informações, sem mencionar a probabilidade de haver erros humanos. Com a automação do processo, pode haver o problema de alto custo de criação e manutenção dos softwares responsáveis pela recuperação de informação estruturada.

O projeto de sistemas de recuperação estruturada na Web necessitam de conhecimentos específicos por parte dos desenvolvedores que os criam, o que pode levar a um custo elevado do projeto.

O objetivo deste trabalho é a criação de uma ferramenta que gere sistemas de recuperação estruturada na Web através da compilação de templates para \gls{crawling} em XML no formato \gls{wpt} para a linguagem Python, além de contribuir com o Scrapy, um projeto de código aberto.

O padrão XML utilizado é o \gls{wpt} \cite{wpt}. O principal motivo para utilizá-lo é fazer uso de padrões Web já estabelecidos e documentados que descrevam a estrutura de páginas HTML.

Com isso, o XML se torna uma \gls{DomainSpecificLanguage} que é dedicada para o domínio de \glspl{webscraper}. O uso de \glspl{DomainSpecificLanguage} faz com que não seja estritamente necessário codificar em uma linguagem de domínio geral, o que torna a tarefa específica de criação e manutenção de \glspl{webscraper} mais fácil e produtiva.

O uso de XML como meio para geração destes sistemas faz com que não haja uma dependência de uma única linguagem de programação, como a linguagem Python utilizada neste trabalho. Qualquer outra linguagem com as devidas bibliotecas pode interpretar XML e gerar código para qualquer outra linguagem de programação. Com isso, há como diminuir o nível de acoplamento entre a solução desenvolvida e a linguagem utilizada, de forma que todo o código em XML possa ser reaproveitado para geração de outros sistemas em outras linguagens de programação sem maiores problemas.

O shell interativo (\gls{repl}) disponível na distribuição da linguagem Python permite a criação e execução instantânea de código, sem que sejam necessários passos como criação de arquivo com o código-fonte, \emph{linkagem} e compilação.

Há também um histórico da linguagem Python para tarefas de recuperação de informação, como o motor de busca do Google, que possui vários componentes escritos em Python \cite{google}.

O framework para \glspl{webscraper} Scrapy, escrito em Python, é um dos poucos disponíveis no mercado. Ele permite a criação de projetos de \glspl{webscraper} profissionais, com uma maior produtividade e organização. A arquitetura do Scrapy permite que o projeto ganhe uma maior escala facilmente, tanto no âmbito de tamanho de projeto quanto na escalabilidade de recursos computacionais. Ele é construído usando o Twisted \cite{twisted}, uma biblioteca de comunicação para redes e orientado a eventos, que possibilita a realização de operações de entrada/saída não bloqueantes e, consequentemente, um modelo de concorrência sem uso de threads. 

A elaboração do presente trabalho se deu em 3 etapas distintas:

\begin{enumerate}
	\item Estudo de viabilidade do \gls{wpt} como linguagem específica de domínio para \glspl{webscraper}, assim como busca de melhores alternativas.
	\item Estudo do funcionamento e arquitetura do \emph{framework} Scrapy.
	\item Implementação de uma ferramenta e seus respectivos testes automatizados para compilação do WPT em um \glspl{spider} do Scrapy.
\end{enumerate}

O código fonte resultado deste trabalho está disponível em seu repositório \emph{online} no endereço  \texttt{http://github.com/herberthamaral/scrapy/} no \emph{branch} "crawling-template".