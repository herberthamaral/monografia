% introdução.tex
\chapter{Introdução}

A World Wide Web (WWW) é um grande repositório de documentos contendo informações sobre as mais diversas áreas do conhecimento. Com a criação de motores de busca de propósito geral, como o Google, buscar informações na Web se tornou uma tarefa trivial e cotidiana.

Porém, mesmo com seus grandes poderes de busca, os motores de busca de propósito geral possuem uma capacidade limitada de recuperação de informações específicas. Em uma busca por "Altura do Michael Jordan em centímetros" ou "Lista no formato XML dos candidatos à deputado federal em 1998", o Google, por exemplo, retornará várias páginas que possam conter a informação, não a(s) informação(ões) em si.

Os motores de busca de propósito específico são desenvolvidos para sanar a limitação de obtenção de dados específicos e de forma estruturada que os motores de busca de propósito geral possuem.

As informações obtidas por motores de busca de propósito específico na Web podem ser úteis para várias organizações. Por exemplo, um determinado sistema de informações de uma empresa de transportes precisa armazenar as infrações cometidas pelos seus motoristas. No entanto, estas informações só estão disponíveis mediante consulta no sítio da Web da autoridade responsável pelo trânsito. Um motor de busca de propósito específico pode ser criado para realizar estas consultas e alimentar o banco de dados do sistema, sem necessidade de intervenção manual do sistema.

Porém, o projeto destes sistemas necessitam de conhecimentos específicos por parte dos desenvolvedores que os criam, o que pode levar a um custo elevado do projeto.

O objetivo deste trabalho é a criação de um software que gera sistemas de recuperação estruturada na Web através da compilação de templates para \emph{crawling} em XML no formato WPT (\emph{Website Parse Template}).

Os resultados mostram que a quantidade de código necessária para criação destes sistemas pôde ser diminuída para 20\% do código original na linguagem de programação Python, além de tornar a criação de sistemas de recuperação estruturada na Web uma tarefa mais fácil através do uso de uma linguagem específica de domínio (\emph{Domain Specific Languages}) ao invés de utilizar primariamente uma linguagem de propósito geral (Python).

O código fonte resultado deste trabalho está disponível em seu repositório \emph{online} no endereço \texttt{http://github.com/herberthamaral/scrapy/}.