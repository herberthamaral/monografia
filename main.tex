\documentclass[a4paper,12pt]{report}

\usepackage{cite}
\usepackage[utf8]{inputenc}
\usepackage{makeidx}
\usepackage{a4wide}
\usepackage{setspace}
\usepackage{graphicx}
\usepackage{listing}
\usepackage{listings}
\usepackage{pslatex}
\usepackage{sectsty}
\usepackage{titlesec}
\usepackage[left=3cm,top=3cm,right=2cm,bottom=2cm]{geometry}


\usepackage{fancyhdr}
\pagestyle{fancy}
\fancyhead{}
\fancyfoot{}
\fancyfoot[FR]{\thepage}
\renewcommand{\headrulewidth}{0pt}
\renewcommand{\contentsname}{Sumário}
\renewcommand{\chaptername}{Capítulo}
\renewcommand{\figurename}{Figura}
\renewcommand{\bibname}{Bibliografia}
\renewcommand{\lstlistingname}{Listagem}
\renewcommand{\appendixname}{Apêndice}
\renewcommand{\listfigurename}{Lista de Figuras}
\renewcommand{\listlistingname}{Lista de Listagens}
\allsectionsfont{\large}
\chapterfont{\Large}


\makeindex
\begin{document}

\begin{titlepage}
\begin{center}
UNIVERSIDADE ESTADUAL DE MONTES CLAROS

Centro de Ciências Exatas e Tecnológicas

Departamento de Ciências da Computação

Curso de Sistemas de Informação
\\[2cm]
Herberth Giuliano Amaral Silva
\\[7cm]
\textbf {Uso de Uma Linguagem Específica de Domínio Baseada em XML Para Geração de Sistemas de Recuperação Estruturadas na Web na Linguagem Python}

\vfill
Montes Claros - MG

Dezembro - 2010



\end{center}
\end{titlepage}

%folha de rosto
\addtocounter{page}{-1}
\thispagestyle{empty}

\begin{center}
	Herberth Giuliano Amaral Silva
	\\[4cm]
	\textbf{Uso de Uma Linguagem Específica de Domínio Baseada em XML Para Geração de Sistemas de Recuperação Estruturadas na Web na Linguagem Python}
\end{center}
	\ \\[3cm]
	
\begin{flushright}
	\begin{small}
		\parbox{200pt}{Projeto orientado de conclusão de curso apresentado ao Curso de Sistemas de Informação,	da Universidade Estadual de Montes Claros como exigência para obtenção do grau de Bacharel	em Sistemas de Informação.}

		\parbox{200pt}{Orientador: Prof.Renato Dourado Maia.}
		
	\end{small}
\end{flushright}

\begin{center}	
	\vfill
	Montes Claros - MG
	
	Dezembro - 2010 
\end{center}

%resumo.tex
\renewcommand{\abstractname}{Resumo}
\begin{abstract}
	Os motores de busca de propósito geral, como o Google, estão presentes no cotidiano de várias pessoas e são uma maneira fácil de buscar por conteúdo, solução de problemas ou até mesmo para fins de tradução. No entanto, estes motores de busca apresentam a limitação de não fornecer meios de recuperação de informações específicas e em formato estruturado. Para estes casos, há os motores de busca de propósito específico, que são projetados especialmente para atender este tipo de necessidade. No entanto, tais projetos podem ser custosos, seja por escassez de mão de obra qualificada ou pela natureza desafiadora do projeto. Este trabalho visa facilitar a criação e manutenção de sistemas de recuperação estruturada na Web através do uso de uma linguagem específica de domínio baseada em XML, utilizada para geração de código de \emph{web scrapers} em Python. Os resultados mostram o ganho de produtividade que um programador, mensurado em linhas de código escritas, tem ao utilizar a ferramenta desenvolvida neste trabalho pode chegar à mais de 30\%. Outros resultados qualitativos discutem o ganho de produtividade ao utilizar uma linguagem específica de domínio, em que o programador não precisa preocupar-se com todos os detalhes inerentes à uma linguagem de propósito geral.
	
	\textbf{Palavras-chave:} scraper, web, python, xml, wpt, dsl, recuperação de informação
\end{abstract}

\renewcommand{\abstractname}{Abstract}
\begin{abstract}
	General purpose search engines, like Google, are present in daily people's life and they are a easy way to search content, problem solving and even for translation needs. However, this kind of search engine has a limitation in not offering ways to retrieve specific information in a structured format. For these cases, there are specific purpose search engines, which are designed specially to attend this kind of need. But these projects can be expensive because of specialized programmer lacking or by the project's challenging nature. This work aims to facilitate the structured web information retrieval sytems creation and maintenance by using a XML domain specific language for web scraper code generation in Python. The quantitative results shows an 30\% productive gain, in terms of number of lines coded when the programmers use the tool developed in this work. Another qualitative results discuss about the productivity gain when the programmer use a domain specific language and do not need to worry about all details present in a general purpose language.

	\textbf{Keywords:} scraper, web, python, xml, wpt, dsl, information retrieval
\end{abstract}


%atalho par o underscore
\def\us{\char`\_}

\thispagestyle{fancy}
\addcontentsline{toc}{chapter}{Sumário}
\tableofcontents
\thispagestyle{fancy}
\listoffigures
\addcontentsline{toc}{chapter}{Lista de figuras}
\thispagestyle{fancy}
\listoflistings
\addcontentsline{toc}{chapter}{Lista de listagens}
\thispagestyle{fancy}

\doublespacing
% introdução.tex
\chapter{Introdução}
\thispagestyle{fancy}

A \gls{www} é um grande repositório de documentos contendo informações sobre as mais diversas áreas do conhecimento. Com a advento de motores de busca de propósito geral, como o Google, buscar informações na Web se tornou uma tarefa trivial e cotidiana.

Porém, mesmo com seus grandes poderes de busca, os motores de busca de propósito geral possuem uma capacidade limitada de recuperação de informações específicas. Em uma busca por "Altura do Michael Jordan em centímetros" ou "Lista no formato XML dos candidatos à deputado federal em 1998", o Google, por exemplo, retornará uma lista de referências de várias páginas na Web que possam vir a conter a informação, sem apresentar a informação no formato que o usuário deseja (como por exemplo, em formato XML).

Os motores de busca de propósito específico são desenvolvidos para sanar a limitação de obtenção de dados específicos e de forma estruturada que os motores de busca de propósito geral possuem.

As informações obtidas por motores de busca de propósito específico na Web podem ser úteis para várias organizações. Por exemplo, um determinado sistema de informações de uma empresa de transportes precisa armazenar as infrações cometidas pelos seus motoristas. No entanto, estas informações só estão disponíveis mediante consulta no sítio da Web da autoridade responsável pelo trânsito. Um motor de busca de propósito específico pode ser criado para realizar essas consultas e alimentar o banco de dados do sistema, sem necessidade de intervenção manual no sistema.

% ------------------------------

A recuperação de Informações na Web é uma tarefa fácil, porém trabalhosa e propensa a erros se feita manualmente por humanos. Fácil, pois humanos conseguem detectar padrões de informações, categorizá-las e armazená-las como quiserem. Trabalhosa, pois podem haver inúmeras páginas com grandes quantidades de informação, o que torna o processo de recuperação manual de informações caro, trabalhoso, tedioso e pouco confável em alguns casos. Dependendo do número de motoristas e da frequência de atualizações das bases de dados locais, a empresa citada no exemplo anterior precisaria contratar algumas pessoas a mais somente para realizar esse tipo de tarefa.

Soluções computacionais como \glspl{webcrawler} ou \glspl{webscraper} fazem parte dos motores de busca de propósito específico  e ajudam na automação do processo de recuperação de informações estruturadas. No entanto, por serem soluções computacionais, demandam que sejam programadas e, geralmente, para um fim específico. Um \gls{webscraper} que foi originalmente desenvolvido para extrair informações de um determinado tipo de página não conseguirá extrair as mesmas informações se a estrutura e/ou organização da página mudarem.

Na prática, um \gls{webscraper} projetado para um sítio precisará sofrer modificações, de forma que consiga extrair informações caso a estrutura da página em \gls{html} mude, seja por uma mudança na estrutura de páginas do sítio ou pelo uso de um novo sítio que possa prover as mesmas informações.

Com a recuperação de informações de forma manual, há o problema da falta de automação de processo, o que pode levar a um alto custo  de recuperação de informações, sem mencionar a probabilidade de haver erros humanos. Com a automação do processo, pode haver o problema de alto custo de criação e manutenção dos softwares responsáveis pela recuperação de informação estruturada.

O projeto de sistemas de recuperação estruturada na Web necessitam de conhecimentos específicos por parte dos desenvolvedores que os criam, o que pode levar a um custo elevado do projeto.

O objetivo deste trabalho é a criação de uma ferramenta que gere sistemas de recuperação estruturada na Web através da compilação de templates para \gls{crawling} em XML no formato \gls{wpt} para a linguagem Python, além de contribuir com um projeto de código aberto.

O padrão XML utilizado é o \gls{wpt} \cite{wpt}. O principal motivo para utilizá-lo é fazer uso de padrões Web já estabelecidos e documentados que descrevam a estrutura de páginas HTML.

Com isso, o XML se torna uma \gls{DomainSpecificLanguage} que é dedicada para o domínio de \glspl{webscraper}. O uso de \glspl{DomainSpecificLanguage} faz com que não seja estritamente necessário codificar em uma linguagem de domínio geral, o que torna a tarefa específica de criação e manutenção de \glspl{webscraper} mais fácil e produtiva.

O uso de XML como meio para geração destes sistemas faz com que não haja uma dependência de uma única linguagem de programação, como a linguagem Python utilizada neste trabalho. Qualquer outra linguagem com as devidas bibliotecas pode interpretar XML e gerar código para qualquer outra linguagem de programação. Com isso, há como diminuir o nível de acoplamento entre a solução desenvolvida e a linguagem utilizada, de forma que todo o código em XML possa ser reaproveitado para geração de outros sistemas em outras linguagens de programação sem maiores problemas.

A produtividade inerente à linguagem Python, combinado com seu poder de prototipação, permite que \glspl{webscraper}  sejam desenvolvidos e testados mais rapidamente.

O shell interativo (\gls{repl}) disponível na distribuição da linguagem Python permite a criação e execução instantânea de código, sem que sejam necessários passos como criação de arquivo com o código-fonte, \emph{linkagem} e compilação.

Há também um histórico da linguagem Python para tarefas de recuperação de informação, como o motor de busca do Google, que possui vários componentes escritos em Python \cite{google}.

O framework para \glspl{webscraper} Scrapy, escrito em Python, é um dos poucos disponíveis no mercado. Ele permite a criação de projetos de \glspl{webscraper} profissionais, com uma maior produtividade e organização. A arquitetura do Scrapy permite que o projeto ganhe uma maior escala facilmente, tanto no âmbito de tamanho de projeto quanto na escalabilidade de recursos computacionais. Ele é construído usando o Twisted \cite{twisted}, uma biblioteca de comunicação para redes e orientado a eventos, que possibilita a realização de operações de entrada/saída não bloqueantes e, consequentemente, um modelo de concorrência sem uso de threads. 

A elaboração do presente trabalho se deu em 3 etapas distintas:

\begin{enumerate}
	\item Estudo de viabilidade do \gls{wpt} como linguagem específica de domínio para \glspl{webscraper}, assim como busca de melhores alternativas.
	\item Estudo do funcionamento e arquitetura do \emph{framework} Scrapy.
	\item Implementação de uma ferramenta e seus respectivos testes automatizados para compilação do WPT em um \glspl{spider} do Scrapy.
\end{enumerate}

O código fonte resultado deste trabalho está disponível em seu repositório \emph{online} no endereço \texttt{http://github.com/herberthamaral/scrapy/} no \emph{branch} "crawling-template".
% referencial_teorico.tex

\pagebreak
\chapter{Referencial teórico}
\index{Referencial Teórico}
\thispagestyle{fancy}

\section{Recuperação de informação}
\index{Recuperação de informação}

Um \gls{sri} é um sistema capaz de armazenar, obter e dar manutenção em informações \cite[p. 2]{kowalski}. Nesse contexto,informação pode ser composto por texto, imagens, áudio, vídeo e outros objetos multimídia.

Um \gls{sri} consiste em um programa de software que facilita a busca de informação por um usuário. O grau de sucesso de um \gls{sri} é relacionado no quanto o mesmo diminui a burocracia (\gls{overhead}) para o usuário encontrar uma informação. O sucesso de um SRI é muito subjetivo e baseia-se em qual informação o usuário quer e sua predisposição a enfrentar ou não a burocracia do processo \cite[p. 4]{kowalski}

\subsection{SRIs para Web}
\index{Recuperação de informação!SRIs para Web}

Comparado com SRIs clássicos, os SRIs para Web encara um conjunto de dados totalmente diferentes \cite[p. 2]{surveyir}. Segundo \cite{surveyir}, dentre os motivos que tornam únicos os SRIs para Web, pode-se citar:

\begin{itemize}
	\item \textbf{Internet Dinâmica} - A Internet muda a cada dia enquanto a maioria dos SRIs clássicos é feita para bancos de dados estáticos.
	\item \textbf{Heterogenidade} - A Internet contém uma grande variedade de tipos de documentos: figuras, arquivos de áudio, textos, scripts, etc.
	\item \textbf{Variedade de línguas} - O número de línguas diferentes na Internet passa de 100.
	\item \textbf{Duplicação} - A cópia é uma outra característica importante da Web: há estimativas de que a quantidade de conteúdo copiado chega a 30\% \cite{surveyir}.
	\item \textbf{Alto número de links} - Em média, cada documento tem mais que 8 links para outras páginas.
	\item \textbf{Consultas heterogêneas} - SRIs para Web têm que lidar com pequenas consultas e não necessariamente bem representadas dos usários de Internet.
	\item \textbf{Grande variância de usuários} - Cada usuário da Web é diferente em vista das suas necessidades, resultados esperados e conhecimento.
	\item \textbf{Comportamento específicos} - É estimado que cerca de 85\% consideram somente a primeira página dos resultados retornados dos motores de busca. 78\% dos usuários nunca modificam seu primeiro termo de consulta.
\end{itemize}

Sendo assim, é possível concluir que o grande desafio de SRIs para Web é suprir as necessidades de informação dos usuários dado que há grande heterogenidade na Web.

\subsection{Motores de busca de propósito geral}
\index{Recuperação de informação!SRIs para Web!Motores de busca de propósito geral}
%verificar se está numerando as subseções

Dentre os motores de busca de propósito geral mais utilizados, há o Google, o Bing e o AltaVista. Dentre os principais objetivos de um motor de busca de propósito geral destacam-se: 

\begin{itemize}
	\item Resultados de qualidade, independentemente da consulta utilizada.
	\item Uma boa cobertura dos documentos disponíveis na Web.
	\item Facilidade de uso.
	\item Ser flexível quanto ao tipo de mídia e documento a serem pesquisados.
\end{itemize}

Todos os serviços citados anteriormente estão de acordo com as necessidades supracitadas. Devido à natureza destes motores de busca, é difícil buscar por somente uma informação específica, como "Altura em centímetros do Michael Jordan". O máximo que esses motores podem fazer é retornar uma página que possa conter essa informação, não a informação em si.


\subsection{Motores de busca de domínio específico}
\index{Recuperação de informação!SRIs para Web!Motores de busca de domínio específico}

Ao contrário dos motores de busca de propósito geral, que buscam informações em sites sem distinção, os motores de busca de domínio específico buscam por informações específicas em sites específicos. Tais motores de busca podem ser utilizados para suprir a necessidade de precisão que se perde com os motores de busca de propósito geral, fornecendo ao usuário resultados mais completos, relevantes e/ou específicos.

Motores de busca de domínio específico estão se tornando populares porque oferecem uma melhor precisão com recursos extras que não são possíveis com motores de busca de propósito geral \cite[p. 8]{wober}.

Um outro problema dos grandes motores de busca é sua falha em manter índices atualizados, devido ao fato de haver um vasto número de páginas que precisam ser indexadas. Um motor de busca de propósito específico possui um conjunto menor de páginas para indexar e consequentemente pode atualizá-las mais frequentemente, o que leva a resultados mais atualizados para os usuários.

A funcionalidade extra se refere à apresentação de interfaces mais sofisticadas para usuários que são configurados para um domínio particular de busca. Desde que todos os usuários visitando uma busca de domínio específico pertecem a um grupo de pessoas com interesses e desejos similares, a busca específica por domínio pode utilizar mais efetivamente o perfil do usuário para dar recomendações mais precisas quando o mesmo tiver dificuldades de especificar suas necessidades \cite[p. 9]{wober}.
% não ficou bom a partir de "Desde"

\subsection{Busca semântica}

A Web semântica é uma extensão da web atual em que a informação dada possui significado bem definido, permitindo que os computadores e pessoas possam trabalhar em cooperação \cite[p. 13]{wober}. Esse novo formato da \gls{www} tem como visão ser tão capaz  quanto a \gls{www} atual, mas consiste em dados legíveis por máquina, que possui o potencial para ter um maior impacto na busca na web.


Na prática, o ''significado bem definido das informações'' é alcançado com o uso de metadados nas páginas Web. \emph{Tags} HTML como \texttt{<title>} e \texttt{<meta>} possuem grande valor no processo de definição de significado da página. Outros fatores como URLs semânticas auxiliam nesse processo também.

Apesar do RDF e do XML estarem se tornando amplamente reconhecidos como veículo padrão para a descrição de metadados, uma enorme parcela de dados semânticos ainda reside dentro de documentos HTML \cite[p. 14]{wober}.

\pagebreak
\section{Python}
\index{Python}

\hyphenation{a-tri-bu-tos}

Segundo \cite{pythondoc}, o Python é uma linguagem poderosa, de propósitos gerais, fácil de aprender e programar, interpretada, orientada a objetos e com alguns outros atributos que a tornam uma linguagem ideal para \emph{scripting} e desenvolvimento rápido de aplicações.

\subsection{Histórico}
\index{Python!Historico}

A linguagem Python foi criada no início dos anos 1990 por Guido Van Rossum \cite{pythonlicense}, no \emph{Stichting Mathematisch Centrum}, na Holanda, com o intuito de ser uma linguagem substituta ao ABC, que apresentava uma série de problemas, especialmente com extensibilidade \cite{pythonfaq}. A linguagem Python, inicialmente, foi criada para ser uma substituta da linguagem ABC, mantendo os poderes de sua API.

Van Rossum permanece como o principal autor da linguagem Python, apesar de receber várias contribuições de colaboradores externos.

Um erro comum é associar a origem o nome da linguagem Python com o nome da serpente. De fato, Van Rossum se inspirou num grupo de comediantes britânicos, o Monty Python, para dar nome à linguagem \cite{pythonfaq} .

\subsection{Características}
\index{Python!Caracteristicas}
Qualidade de software, produtividade do programador, portabilidade, bibliotecas de suporte, integração de componentes e diversão são os principais motivos do uso da linguagem \cite[p. 48,49]{learningpython}. 

Em seu modelo, a linguagem Python implementa uma sintaxe deliberadamente simples e legível e um modelo de programação coerente \cite[p. 50]{learningpython}. 

Dentre as razões históricas, a linguagem Python foi concebida no início da década de 1990, quando ocorreu o \emph{boom} da Internet e quando o número de programadores se tornou escasso para a demanda de software. Enquanto linguagens de baixo nível como Assembly ou C priorizam a \emph{produtividade de máquina}, Python prioriza a \emph{produtividade do programador}. 

Python é otimizada para produtividade do programador. Recursos como sintaxe simples, tipagem dinâmica, não necessidade de compilação e um conjunto de ferramentas embutidas permitem que programadores desenvolvam programas em uma fração de tempo necessária em comparação a quando usam outras ferramentas \cite[p. 50]{learningpython}. 

Entretanto, existe um \gls{tradeoff} no uso da linguagem Python: a velocidade de execução. Programadores tendem a ser mais produtivos com o uso de linguagens com os atributos que Python possui. Porém, o código Python não é um código de máquina, portanto o mesmo precisa ser interpretado a cada execução, o que o torna mais lento. 

Outros fatores como a tipagem dinâmica tendem a trazer mais \gls{overhead} na execução dos programas em Python, degradando ainda mais sua velocidade de execução, e seu uso difícil ou inviável para projetos que dependam estritamente de velocidade de execução (ex: componentes de baixo nível, como Kernels e drivers de dispositivos, aplicativos de produtividade, como suítes de escritórios e CAD e outros softwares de grande porte).

\subsubsection{Tipagem dinâmica}
\index{Python!Ambientes e plataformas!Tipagem Dinâmica}

A tipagem dinâmica é uma propriedade de uma linguagem de programação que permite que a checagem de tipo seja feito em tempo de execução, ao contrário da tipagem estática, onde a verificação de tipo é feita em tempo de compilação.


Segundo \cite{dynamic_langs}, as principais vantagens ao se utilizar uma linguagem de tipagem dinâmica é evitar a rigidez de linguagens estaticamente tipadas, tornar mais fácil a prototipação de sistemas que possuem requisitos ainda não conhecidos ou que mudam de uma forma não previsível, aumentar o reuso de código, diminuir a verbosidade e o custo do projeto, sem necessariamente ser menos seguras que as linguagens de tipagem estática.

Devido ao seu baixo nível de burocracia e alto poder de prototipação, as linguagens dinâmicas são freqüentemente para integração entre componentes \cite{scripting}. Por esse motivo, as linguagens dinâmicas também são conhecidas por \emph{linguagens-cola} ou \emph{linguagens de integração de sistemas}.

Uma linguagem que não é apegada a tipos permite que a integração entre componentes seja bem mais fácil, pois não há restrições primárias sobre como os retornos de cada componente devam ser utilizados: todos eles são apresentados de uma única forma \cite{scripting}.

Linguagens dinâmicas, como o Python, são ideais para pesquisas e provas de conceito, graças a seu alto poder de prototipação, o que ajuda a explicar o quão Python está em uso na comunidade científica, como demonstrado em \cite{python_scientific_world}.

\subsection{Ambientes e plataformas}
\index{Python!Ambientes e plataformas}

Python é uma linguagem portável e multiplataforma. Isso significa que um código em Python pode ser executado nos mais diversos ambientes e sistemas operacionais, como Windows, Linux, Mac OS e em até sistemas operacionais móveis como Symbian e Android.

Há também a possibilidade de executar a linguagem Python em outros ambientes diferentes do original (CPython). Iniciativas como Jython (Python para a \emph{Java Virtual Machine}) e o IronPython (Python para o ambiente Microsoft.NET) permitem que o Python seja executado dentro das duas das maiores plataformas de desenvolvimento de software da atualidade, aproveitando seus recursos e suas funcionalidades. Dessa forma, um programa em Python pode utilizar o Swing\footnote{Swing é uma interface de programação de aplicativos Java para interfaces gráficas} através do Jython para o desenvolvimento de uma interface gráfica em ambiente Java ou pode utilizar o \emph{Windows Communication Foundation} como \emph{framework} de troca de mensagens no ambiente Microsoft.NET.

\subsubsection{CPython}
\index{Python!Ambientes e plataformas!IronPython}

O CPython é a implementação padrão da linguagem Python e é escrita na linguagem de programação C \cite[p.6]{pypy}. A linguagem é implementada por um compilador que traduz código Python em um código código-objeto de altíssimo nível (\emph{very high level}) e por uma máquina virtual que interpreta o código.

\subsubsection{IronPython}
\index{Python!Ambientes e plataformas!IronPython}

O IronPython \cite{ironpython} é uma implementação da linguagem de programação Python que é executada no framework .NET e Silverlight. Suporta um \gls{shell} interativo (como a maioria das implementações da linguagem Python) com compilação dinâmica. É integrado com o resto do framework .NET e torna todas as bibliotecas do .NET disponíveis para programadores Python, enquanto mantém a compatibilidade com a linguagem Python.

\subsubsection{Jython}
\index{Python!Ambientes e plataformas!Jython}

Jython \cite{jython} é uma implementação da linguagem Python para a JVM (\emph{Java Virtual Machine}). O Jython torna possível a execução da sintaxe da linguagem de programação Python na plataforma Java, o que permite uma integração transparente com as bibliotecas Java e outras aplicações baseadas em Java. 

\subsubsection{PyPy}
\index{Python!Ambientes e plataformas!PyPy}

PyPy é uma implementação do Python escrita em Python \cite[p. 7]{pypy}. A idéia principal é escrever uma especificação de alto nível do interpretador em um subtrato restrito do Python chamado RPython (\emph{Restricted Python}) com o intuito de ser traduzido para executáveis eficientes de baixo nível para o ambiente C/POSIX, \gls{jvm} e \Gls{cli}, o que garante a portabilidade. 

\subsection{Python e Computação Científica}
\index{Python!Python e Computação Científica}

Segundo \cite{python_scientific_world}, Python e estensões como o NumPy \cite{numpy} estão se tornando padrão para muitas ciências que precisam processar grande quantidades de dados, desde a neuroimagem à astronomia.

Conferências anuais especificamente voltadas para usos de Python em computação científica, como o SciPy \cite{scipy} (tanto na versão norte-americana quanto na européia) vem ganhando mais representatividade a cada ano.

Ambientes de desenvolvimento integrado como o PythonXY \cite{pythonxy} ajudam a mostrar para cientistas que a linguagem Python com seus pacotes de bibliotecas e softwares para computação científica pode ser uma excelente alternativa para ambientes como o Matlab \footnote{\url{http://www.mathworks.com/products/matlab/}}. 

Um dos casos mais famosos da linguagem Python para os sistemas de recuperação de informações é o Google\cite{google}. Ela representa um papel fundamental dentro da estrutura de motores de busca do Google, sendo responsável pelos seus \emph{crawlers} e pelos seus servidores de URL \cite{surveyir}.

\pagebreak
\section{Scrapy}
\index{Scrapy}

Scrapy é um framework rápido e de alto nível para \gls{screenscraping} e \gls{crawling}, usado para navegar por websites e extrair dados estruturados de suas páginas. Pode ser usado para um grande número de propósitos, desde mineração de dados a monitoramento e automação de testes \cite{scrapy}. Mesmo que o Scrapy tenha sido originalmente feito para \gls{screenscraping}, ele também pode ser utilizado para extrair dados usando APIs (como as APIs de serviços da Amazon\footnote{\url{http://www.amazon.com/webservices}}) ou pode ser utilizado como um \gls{webcrawler} de propósito geral.

Dentre as principais funcionalidades que o Scrapy apresenta, podem-se citar:

\begin{itemize}
	\item Suporte embutido para selecionar e extrair dados de fontes em HTML ou XML.
	\item Suporte embutido para limpeza e sanitização dos dados obtidos utilizando uma coleção de filtros reutilizáveis (chamados de \emph{Item Loaders}) compartilhados entre todos os \glspl{webcrawler}.
	\item Suporte embutido para geração e exportação de \glspl{feed} em múltiplos formatos e armazenamento dos mesmos (seja em FTP ou localmente).
	\item Um gerenciador de mídia para download automático de imagens (ou qualquer outro tipo de mídia) associado com os itens obtidos.
	\item Suporte para extensão do Scrapy, sendo possível adicionar funcionalidades ao framework através do uso de sinais e uma API bem definida.
	\item Grande quantidade de \glspl{middleware} e extensões para gerenciamento de cookies e sessões, compressão HTTP, autenticação HTTP, cache HTTP, mudança do header \emph{user-agent} \footnote{\Gls{httpheader}} que identifica o navegador do cliente e restrições sobre a profundidade de \gls{crawling}.
	\item Suporte robusto para auto-detecção de codificação, assim como manipulação de codificações estrangeiras, quebradas e fora dos padrões.
	\item Extensa coleção de status para métricas do \gls{webcrawler}, úteis para monitoramento de performance e de disponibilidade.
	\item Um console interativo para teste de \glspl{xpath}, útil para \gls{debugging}.
	\item Um serviço a nível de sistema feito para facilitar a implantação e execução dos crawlers em produção.
	\item \Glspl{webservice} e console telnet embutidos para monitoração, controle, introspecção e \gls{debugging} do crawler.
\end{itemize}

\subsection{Arquitetura}
\index{Scrapy!Arquitetura}

O diagrama demonstrado na figura \ref{scrapy_architecture} apresenta uma visão geral da arquitetura do Scrapy com seus componentes e o fluxo de dados que há dentro do sistema (mostrado em setas verdes). A seguir, cada um dos componentes, bem como o fluxo de dados serão descritos.

\begin{figure} [ht]
	\centering
	\includegraphics[scale=1]{scrapy_architecture.png}
	\caption{Visão geral da arquitetura do scrapy \cite{scrapy_arch}}
	\label{scrapy_architecture}
\end{figure}

\subsubsection{\emph{Scrapy engine}}
\index{Scrapy!Arquitetura!Scrapy engine}

O \emph{engine} é responsável por controlar o fluxo de dados entre todos os componentes do sistema, e disparar eventos quando certas ações ocorrerem.

\subsubsection{\emph{Scheduler} (agendador)}
\index{Scrapy!Arquitetura!Scheduler}

O \emph{scheduler} (agendador) recebe requisições do \emph{engine} e as enfileira para a realimentação da engine no futuro.

\subsubsection{\emph{Downloader}}
\index{Scrapy!Arquitetura!Downloader}

O \emph{Downloader} é responsável por obter páginas da Web e alimentar o \emph{engine}, que por sua vez alimenta os crawlers/spiders.

\subsubsection{Item pipeline}
\index{Scrapy!Arquitetura!Item pipeline}

O \textit{Item Pipeline} é responsável pelo processamento dos itens uma vez que eles foram extraídos pelos \textit{crawlers}/\textit{spiders}. Tarefas típicas do \textit{Item Pipeline} incluem limpeza, validação e persistência do item.

\subsubsection{Spider middlewares}
\index{Scrapy!Arquitetura!Spider middlewares}

\emph{Spider} \glspl{middleware} fica entre a \emph{Engine} e os \emph{Spiders} e podem processar a entrada de um sipder (respostas) e saídas (itens e requisições). Eles fornecem um mecanismo conveniente para estender as funcionalidades do Scrapy através da adição de código customizado.

\subsubsection{Scheduler middlewares}
\index{Scrapy!Arquitetura!Scheduler middlewares}

\emph{Scheduler} \glspl{middleware} fica entre a \emph{Engine} e o \emph{Scheduler} e processa requisições quando as mesmas passam do \textit{engine} para o \textit{Scheduler} e vice-versa. Eles fornecem um mecanismo conveniente para estender as funcionalidades do Scrapy através da adição de código customizado.

\subsubsection{Fluxo de dados}
\index{Scrapy!Arquitetura!Fluxo de dados}

O fluxo de dados do Scrapy é controlado pela Engine e funciona da seguinte forma:

\begin{enumerate}
	\item O \emph{Engine} abre um domínio, localiza o \emph{Spider} que manipula aquele domínio e pede para o \emph{Spider} a primeira URL para navegar.
	\item O \emph{Engine} pega a primeira URL para navegar do Spider e a agenda no Scheduler como Requisições.
	\item O \emph{Engine} pede para o \emph{Scheduler} as próximas URLs para navegar.
	\item O \emph{Scheduler} retorna a próxima URL para navegação para o \emph{Engine} e este a envia para o \emph{Downloader} passando pelo \emph{Downloader Middleware}.
	\item Uma vez que o download da página terminou, o Downloader gera uma Resposta (\emph{Response}) (com a página) e a manda para o \emph{Engine}, passando pelo \emph{Downloader Middleware}.
	\item O \textit{Engine} recebe a resposta do \textit{Downloader} e a envia para o \textit{Spider} processar, passando pelo \textit{Spider Middleware}.
	\item O \textit{Spider} processa a Resposta (\textit{Response}) e retorna os itens obtidos e novas Requisições(\textit{Requests}) para o \textit{Engine}.
	\item O \textit{Engine} envia os itens obtidos (retornados pelo \textit{spider}) para o \textit{Item Pipeline} e envia as Requisições(retornadas pelo \textit{Spider}) para o \textit{Scheduler}.
	\item O processo se repete a partir do passo 2 até que não existam mais Requisições no \textit{Scheduler} e o \textit{Engine} fecha o domínio.
\end{enumerate}

\subsubsection{E/S orientado a eventos}
\index{Scrapy!Arquitetura!Networking orientado a eventos}

O Scrapy é escrito utilizando o Twisted \cite{twisted}, um \textit{framework} popular para programação orientada a eventos para Python. Desta forma, ele é implementado utilizado código não-bloqueante (ou assíncrono) para aumentar a concorrência.

\subsection{Serviços embutidos}
\index{Scrapy!Serviços embutidos}

O Scrapy possui uma gama de serviços embutidos implementados para facilitar a monitoração e a manutenção do sistema. A seguir, são apresentados cada um destes serviços embutidos.

\subsubsection{\textit{Logging}}
\index{Scrapy!Serviços embutidos!Logging}

O \textit{log} é um histórico de execução do sistema. Através da análise de logs, é possível detectar e achar a fonte de problemas. O Scrapy possui um componente de \textit{logging} que grava as mensagens de log em um arquivo em disco e possui cinco níveis de log \cite{scrapy_log}:

\begin{enumerate}
	\item \texttt{\textbf{CRITICAL}} - Para erros críticos
	\item \texttt{\textbf{ERROR}} - Para erros regulares
	\item \texttt{\textbf{WARNING}} - Para avisos
	\item \texttt{\textbf{INFO}} - Para mensagens informativas
	\item \texttt{\textbf{DEBUG}} - Para mensagens de debug
\end{enumerate}

O nível de log é configurável e um determinado nível irá catalogar todas as mensagens do nível mais as mensagens dos níveis abaixo. Por exemplo, se o nível de log for configurado para 5 (\texttt{DEBUG}), o sistema irá gravar todas as mensagens de log, ao passo que se o nível de log for configurado para 2 (\texttt{ERROR}), somente as mensagens de erros regulares e erros críticos serão gravadas.

\subsection{Código fonte}

O Scrapy é um \emph{framework} de código aberto, disponível em um repositório no Github (\url{http://github.com/}) no endereço \url{http://github.com/insophia/scrapy/}.

\pagebreak
\section{Git e Github}

Git é um sistema de controle de versões distribuído escrito inicialmente por Linus Torvalds, criador do kernel do Linux e atualmente mantido por Junio Hamano \cite{progit}.

\subsection{Controle de versões}

Um controle de versões é um sistema que guarda modificações a um arquivo ou conjunto de arquivos com o passar do tempo, com o principal intuito de obter versões específicas a qualquer momento \cite[p. 1]{progit}.


\subsubsection{Controle local de versões}

Segundo \cite[p.2]{progit}, é comum pessoas controlarem versões manualmente através da cópia de arquivos para outras pastas, provavelmente com alguma indicação de data, como ilustrado na Figura \ref{local_version_control}. Apesar de simples, esse sistema de versionamento é propenso a erros devido ao fato que se vale da memória humana para encontrar versões específicas e controlar alterações.

\begin{figure} [ht]
	\centering
	\includegraphics[scale=0.4]{local_version_control.png}
	\caption{Diagrama de um controle de versões locais \cite[p. 1]{progit}}
	\label{local_version_control}
\end{figure}

Com o passar do tempo, pequenos e simples sistemas de controle de versão com banco de dados locais para guardar listas de alterações foram criados para ajudar a resolver esse problema.

\subsubsection{Controle de versões centralizado}

O maior problema com controle de versões locais é a dificuldade de colaboração \cite[p. 3]{progit}. Para lidar com esse problema, sistemas de controle de versão centralizados, ilustrados na Figura \ref{central_version_control}, foram criados. Esses sistemas, como o CVS, Subversion e Perforce, possuem um único servidor que contém todos os arquivos versionados e um número de clientes que acessam os arquivos de um repositório central.

\begin{figure} [ht]
	\centering
	\includegraphics[scale=0.4]{central_version_control.png}
	\caption{Diagrama de um controle de versões locais \cite[p. 2]{progit}}
	\label{central_version_control}
\end{figure}

Apesar de permitir a colaboração, há várias desvantagens com esta abordagem, principalmente no que diz respeito ao acesso: se o servidor que guarda estas informações sofrer um problema, ninguém poderá colaborar em um projeto ou salvar as últimas alterações (\emph{commit}) feitas.

\subsubsection{Controle de versões distribuído}

Os sistemas de controle de versões distribuídos, ilustrados na Figura \ref{dist_version_control}, ao invés de apenas obter apenas as últimas versões dos arquivos de um servidor central, uma cópia completa do repositório é copiada para a máquina local \cite[p. 4]{progit}. Dentre exemplos de sistemas de controle de versão distribuídos, pode-se citar o Git, Mercurial, Bazaar e Darcs.

\begin{figure} [ht]
	\centering
	\includegraphics[scale=0.4]{dist_version_control.png}
	\caption{Diagrama de um controle de versões distribuído \cite[p. 4]{progit}}
	\label{dist_version_control}
\end{figure}

Com sistemas de controle de versão distribuídos podem haver servidores centrais, mas a queda ou desconexão não impede a colaboração, já que, como todos os usuários de um determinado repositório possuem uma cópia completa do repositório localmente, eles podem trocar informações entre si sem maiores problemas \cite[p. 4]{progit}.

\subsection{Uma breve história do Git}

O Git surgiu da necessidade de criar um sistema de controle de versões para substituir o BitKeeper, um sistema de controle de versões distribuídos, devido à problemas com a empresa que o desenvolvia \cite[p. 5]{progit}. Alguns objetivos do novo sistema de controle de versões do kernel foram:

\begin{itemize}
	\item Velocidade.
	\item Design simples.
	\item Forte suporte para desenvolvimento não-linear (milhares de \emph{branches} - Ramificações).
	\item Totalmente distribuído.
	\item Habilidade para gerenciar grandes projetos, como o kernel do Linux, eficientemente. 
\end{itemize}


Segundo \cite{progit}, desde 2005, o Git evoluiu e se tornou maduro para ser fácil, mantendo as qualidades iniciais e sendo incrivelmente rápido, muito eficiente com projetos grandes e possui um sistema de \emph{branching} para desenvolvimento não-linear.

\subsection{Arquitetura do Git}

A seguir, são apresentados os itens mais relevantes no que diz respeito à arquitetura do Git.

\subsubsection{\emph{Snapshots} ao invés de diffs}

Uma das maiores diferenças entre o Git e outros Sistemas de Controle de Versão (SCM), como o Subversion e o Mercurial, é a forma como o Git trata os dados \cite[p. 6]{progit}. Esses sistemas tratam as informações como uma série de mudanças (deltas ou diffs) ao longo do tempo, como ilustrado na Figura \ref{scm_delta}:

\begin{figure} [ht]
	\centering
	\includegraphics[scale=0.5]{scm_delta.png}
	\caption{Diagrama que ilustra como outros sistemas tratam informações ao longo do tempo\cite[p. 6]{progit}}
	\label{scm_delta}
\end{figure}

O Git trata as informações como se fossem \emph{snapshots} de um pequeno sistema de arquivos. Segundo \cite[p. 6]{progit} o Git basicamente "tira uma foto" de como os arquivos estão toda vez que um \emph{commit}, ou o salvamento do estado do projeto no Git.

\pagebreak

\section{\emph{Website Parse Template}}
\index{Website Parse Template}


Website Parse Template (WPT) é um formato aberto baseado em XML que fornece informações adicionais a web crawlers como estrutura e conteúdo do HTML. O WPT é compatível com esquemas XML, como o RDF e o OWL \cite{wpt}. É composto por três seções principais:

\begin{itemize}
	\item \textbf{Templates} - Seção mandatória, que contém descrições sobre estrutura e conteúdo de páginas da web em HTML.
	\item \textbf{Ontologia} - Seção opcional que define conceitos e relações usadas em um website.
	\item \textbf{URLs} - Seção opcional, que associa certos padrões de URL para grupos de páginas web para templates específicos. 
\end{itemize}

Cada uma das seções do WPT apresentadas acima será discutida a seguir.

\subsection{Seções do WPT}
\index{Website Parse Template!Seções do WPT}

\subsubsection{Templates}
\index{Website Parse Template!Seções do WPT}

A seção de templates descreve a estrutura do HTML fazendo referências aos elementos HTML correspondentes da página em específico \cite{wpt}.

O Template inicia com a tag \texttt{<ow:template>} e termina com  \texttt{</ow:template>}. É obrigatório especificar um nome de template único dentro da tag \texttt{<ow:template>} e definir a URL correspondente ao template específico. O Template é composto de blocos correspondentes a cada elemento estrutural de uma página web em específico. Cada template precisa ter ao menos um bloco. Um bloco faz referência ao elemento HTML apropriado através de uma ou muitas combinações dos seguintes métodos de referência: TagID, XPath e Pattern. Cada bloco precisa iniciar com uma tag \texttt{<ow:block>} e fechar com \texttt{</ow:block>}. É necessário indicar elementos HTML específicos dentro da tag de abertura do bloco, como demonstrado na Listagem \ref{wpt_exemplo_template_tres_blocos}.

\pagebreak
\lstset{language=XML,
basicstyle=\scriptsize,
caption={Exemplo de um Template com três blocos \cite{wpt}},
captionpos=b
}
\begin{lstlisting}[label=wpt_exemplo_template_tres_blocos]
<ow:template ow:name="Template Example" ow:url="http://www.example.com/index.php">
  <ow:block ow:tagid="ex1" ow:xpath="/html/body/div/div" ow:pattern="content (object[[a-z]*])">
    <!--descricao do conteudo -->
  </ow:block>
  <ow:block ow:tagid="ex2">
    <!-- descricao do conteudo -->
  </ow:block>
  <ow:block ow:xpath="/html/body/div/div/table/tr[1]/td">
    <!-- descricao do conteudo -->
  </ow:block>
</ow:template>
\end{lstlisting}

Cada bloco contém a descrição do conteúdo dos elementos HTML representados isoladamente ou dentro de outro bloco. Blocos embutidos são usados para descrever elementos HTML específicos ("bloco pai") que incluem um ou mais elementos ("bloco filho"),como demonstrado na Listagem \ref{wpt_blocos_embutidos}.

\lstset{language=XML,
basicstyle=\scriptsize,
caption={Exemplo de um Template com blocos embutidos \cite{wpt}},
captionpos=b
}
\begin{lstlisting}[label=wpt_blocos_embutidos]
<ow:template ow:name="Template Example" ow:url="http://www.example.com/index.php">

  <ow:block ow:xpath="/html/body/div/div">
    <ow:block ow:xpath="/html/body/div/div/table/tbody/tr[1]/td">
      <!-- descricao do conteudo -->
    </ow:block>
  </ow:block>
  
</ow:template>
\end{lstlisting}

A descrição do conteúdo pode ser informada usando conceitos definidos na seção \emph{Ontology} ou qualquer linguagem/formato suportado: RDF, CWL, etc. É necessário declarar \textit{namespaces} dos esquemas XML usados dentro da tag \texttt{<ow:wpt>} e o nome da ontologia dentro da tag \texttt{<ow:template>}, como mostrado na Listagem \ref{wpt_descricao_conteudo}.

\pagebreak
\lstset{language=XML,
basicstyle=\scriptsize,
caption={Exemplo de um Template com instâncias de descrição de conteúdo\cite{wpt}},
captionpos=b
}
\begin{lstlisting}[label=wpt_descricao_conteudo]
<ow:wpt xmlns:ow="http://www.omfica.org/schemas/ow/0.9"
 xmlns:rdf="http://www.w3.org/1999/02/22-rdf-syntax-ns#"
 ow:host="http://www.example.com">

  <ow:template ow:name="Template Example" ow:url="http://www.example.com/index.php"
   ow:ontology="ontology_example">

     <!-- descricao de conteudo utilizando conceitos definidos -->
     <ow:block ow:tagid="ex1" ow:xpath="/html/body/div/div" ow:pattern="Wellcome (user.name[[A-Za-z]*])">
       conceito_de_ontologia
     </ow:block>
     <!-- descricao de conteudo utilizando a sintaxe RDF -->
     <ow:block ow:tagid="ex2">
       <rdf:Description rdf:about="http://www.example.com/index.php">
       </rdf:Description>
     </ow:block>
     <!-- descricao de conteudo utiliznado CWL.unl -->
     <ow:block ow:xpath="/html/body/div/div/table/tr[1]/td">
       {cwl.unl}
       {/cwl.unl}
     </ow:block>
  </ow:template>

</ow:wpt>
\end{lstlisting}

\subsubsection{URLs}
\index{Website Parse Template!URLs}

Essa seção define as URLS/padrões de URLs das páginas web descritas na seção de Templates. É mandatória se os templates não definem as URLs/padrões de URL das páginas web \cite{wpt}.

De acordo com a seção de Templates, essa seção também pode consistir de vários blocos/unidades. Cada um desses blocos iniciam com a tag \texttt{<ow:urls>} e terminam com a tag \texttt{</ow:urls>}, como mostrado na Listagem \ref{wpt_exemplo_padroes_urls}.


\lstset{language=XML,
basicstyle=\scriptsize,
caption={Exemplo de padrões de URLs \cite{wpt}},
captionpos=b,
label="wpt_url_pattern"
}
\begin{lstlisting}[label=wpt_exemplo_padroes_urls]
<ow:urls ow:name="Artist page on Yahoo! Music" ow:template="Artist page on Yahoo! Music">
  <ow:url>http://music.yahoo.com/ar-8206256---Amy-Winehouse</ow:url>
  <ow:url>http://music.yahoo.com/ar-([artist.id[0-9]*])---(artist.name[[A-Za-z0-9-]*])</ow:url>
</urls>
\end{lstlisting}

Especificações utilizando expressões regulares (RegExp) são utilizadas para descrição de padrões de URL. O padrão de URL informado na listagem \ref{wpt_exemplo_padroes_urls} também inclui a real URL representada. Os conceitos necessários para definição de padrões de URL (tais como "id" e "nome") são definidos na seção Ontology.

\subsubsection{\textit{Ontology}}
\index{Website Parse Template!URLs}

A seção \textit{Ontology} contém enumerações e definições de todos os conceitos utilizados no sítio \cite{wpt}. Os conceitos listados precisam ser definidos dentro das tags \texttt{<ow:ontology>} e \texttt{<ow:ontology>}, como mostrado na Listagem \ref{wpt_exemplo_ontology}. É necessário especificar o nome da ontologia dentro da tag \texttt{<ow:ontology>}. O WPT permite tanto o uso de OWL ou da \emph{WPT Ontology language} para definição de conceitos. A principal diferença dentre essas línguas é que a \emph{WPT Ontology language} fornece uma sintaxe simplificada para conceitos e definição de relações.

\lstset{language=XML,
basicstyle=\scriptsize,
caption={Definição conceitual de ''artista'' utilizando a \emph{WPT Ontology language} \cite{wpt}},
captionpos=b,
}
\begin{lstlisting}[label=wpt_exemplo_ontology]
<ow:ontology ow:name="general">
  <ow:concept ow:name="artist">
    <ow:inherit>person</ow:inherit>
    <ow:has>name</ow:has>
    <ow:has>album</ow:has>
    <ow:has>track</ow:has>
    <ow:has>image</ow:has>
    <ow:has>bio</ow:has>
    <ow:has>video</ow:has>
    <ow:has>id</ow:has>
  </ow:concept>
  <ow:concept>logo</ow:concept>
  <ow:concept>menu</ow:concept>
  <ow:concept>advertisement</ow:concept>
</ow:ontology>
\end{lstlisting}

Cada definição de conceito inicia com a tag \texttt{<ow:concept>} e termina com a tag\texttt{</ow:concept>}. A tag \texttt{<ow:inherit>} representa relações de herança e a tag \texttt{<ow:has>} representa relações atribuíveis entre dois conceitos. Os conceitos definidos tem um atributo padrão - identificador de objeto (id) para ser utilizado por \glspl{webcrawler} para coordenar os mesmos atributos do objeto utilizados em páginas diferentes do mesmo website.

% justificativa
\chapter{Justificativa}

Recuperação de Informações na Web é uma tarefa fácil, porém trabalhosa se feita manualmente por humanos. Fácil, pois conseguimos detectar padrões de informações, categoriza-las e armazena-las como quisermos. Trabalhosa, pois podem haver inúmeras páginas com grandes quantidades de informação, o que torna o processo de recuperação manual de informações caro, trabalhoso e tedioso.

Soluções computacionais como \emph{web crawlers} ou \emph{web scrapers} ajudam na automação do processo de recuperação de informações estruturadas. 

No entanto, por serem soluções computacionais, demandam que sejam programadas e, geralmente, para um fim específico. Um \emph{web scraper} que foi originalmente desenvolvido para extrair informações de um determinado tipo de página, não conseguirá extrair as mesmas informações se a estrutura e/ou organização da página mudarem.

Na prática, um \emph{web scraper} projetado para um sítio, precisará sofrer modificações de forma que consiga extrair informações caso a estrutura da página em HTML mude, seja por uma mudança na estrutura de páginas do sítio ou pelo uso de um novo sítio que possa prover as mesmas informações.

Com a recuperação de informações de forma manual, havia o problema da falta de automação de processo, o que pode levar a um alto custo humano de recuperação de informações. Com a automação do processo, pode haver o problema de alto custo de criação e manutenção dos softwares responsáveis pela recuperação de informação estruturada.

O presente trabalho visa criar um método que diminua os custos de criação e manutenção de \emph{web scrapers} por meio da criação automatizada destes softwares através do uso de \emph{templates} em XML.

\section{XML e WPT}

Para o contexto deste trabalho, o XML desempenha um papel fundamental, onde as regras do \emph{web scraper} serão definidas para posterior geração do código do software de recuperação de informações em si.

O formato XML utilizado é o WPT (Website Parse Template)\cite{wpt}. O principal motivo para utiliza-lo é fazer uso de padrões Web já estabelecidos e documentados que descrevam a estrutura de páginas HTML.

Com isso, o XML se torna uma \emph{DSL} (\emph{Domain Specific Language} - Linguagem Específica de um Domínio) que é dedicada para o domínio de \emph{web scraping}. O uso de \emph{DSLs} faz com que não seja estritamente necessário codificar em uma linguagem de domínio geral, o que torna a tarefa específica de criação e manutenção de \emph{web scrapers} mais fácil e produtiva.

\section{Python}

A produtividade inerente à linguagem Python combinado com seu poder de prototipação, permite que \emph{web scrapers} sejam desenvolvidos e testados mais rapidamente.

O shell interativo (\emph{REPL - Read and Eval Print Loop}) disponível na distribuição da linguagem Python permite a criação e execução instantânea de código, sem precisar passar por passos como criação de arquivo com o código-fonte, linkagem e compilação.

Há também um histórico da linguagem Python para tarefas de recuperação de informação, como o motor de busca do Google, que possui vários componentes escritos em Python \cite{google}.

\section{Scrapy}

O framework de \emph{web scraping} Scrapy, escrito em Python, é um dos poucos disponíveis no mercado. Ele permite a criação de projetos de \emph{web scrapers} profissionais, com uma maior produtividade e organização.

A arquitetura do Scrapy permite que o projeto ganhe uma maior escala facilmente, tanto no âmbito de tamanho de projeto quanto na escalabilidade de recursos computacionais. Ele é construído usando o Twisted \cite{twisted}, uma engine de comunicação para redes e orientado a eventos, que possibilita a realização de operações de I/O não bloqueantes e, consequentemente, um modelo de concorrência sem uso de threads. 

Esta característica é vital, pois \emph{web scrapers} dependem de várias operações de I/O em rede para recuperação das páginas Web. Sem os recursos de I/O orientada à eventos, múltiplos processos do Python precisariam ser utilizados para haver concorrência, uma vez que a implementação original da linguagem não possui threads a nível de Kernel \cite{python_threads}.


% objetivos
\chapter{Objetivos}
\thispagestyle{fancy}

\section{Objetivo Geral}

Facilitar a criação de sistemas para recuperação de informações estruturadas na Web através do uso de templates em XML.

\section{Objetivos específicos}

\begin{enumerate}
	\item Diminuir os custos de criação e manutenção de sistemas para recuperação de informação estruturada na Web.
	\item Contribuição com projetos de código aberto.
\end{enumerate}
% metodologia
\chapter{Metodologia}

A elaboração do presente trabalho se deu em 3 etapas distintas:

\begin{enumerate}
	\item Estudo de viabilidade do \emph{Website Parse Template} (WPT) como \emph{Domain Specific Language} (DSL) para \emph{web scrapers}, assim como busca de melhores alternativas.
	\item Estudo do funcionamento e arquitetura do framework Scrapy.
	\item Implementação de um protótipo e seus respectivos testes automatizados para compilação do WPT em um \emph{spider} do Scrapy.
\end{enumerate}

A seguir, é mostrada uma descrição mais detalhada de cada etapa de produção deste trabalho.

\section{Estudo de viabilidade do WPT e outras alternativas}

Uma preocupação constante no desenvolvimento do presente trabalho foi a procura por adoção de padrões abertos, que sejam amplamente utilizados e que não representem um gargalo durante o desenvolvimento.

Além do Website Parse Template, outras duas alternativas foram pesquisadas: o Protocol Buffer \cite{protobuf} e o JSON \cite{JSON}. Ambas foram descartadas pelos motivos citados anteriormente ao passo que o WPT se encaixou bem na lista de requisitos.

\section{Estudo do Scrapy}

Uma das grandes vantagens percebidas do framework Scrapy é a sua abundante documentação, seja a oficial em documentos HTML ou a documentação disponível dentro do próprio código fonte do mesmo.

O Scrapy possui um utilitário de linha de comando que permite a criação de novos projetos de \emph{web scrapers}, assim como sua execução, manutenção e observação de funcionamento do projeto. O exemplo a seguir ilustra a saída da execução do comando "scrapy":


\lstset{basicstyle=\scriptsize,
caption={Saída da execução do comando ''scrapy''},
captionpos=b
}
\begin{lstlisting}
Scrapy 0.11 - no active project

Usage:
  scrapy <command> [options] [args]

Available commands:
  fetch         Fetch a URL using the Scrapy downloader
  runspider     Run a self-contained spider (without creating a project)
  settings      Get settings values
  shell         Interactive scraping console
  startproject  Create new project
  version       Print Scrapy version
  view          Open URL in browser, as seen by Scrapy

Use "scrapy <command> -h" to see more info about a command
\end{lstlisting}

O protótipo desenvolvido neste trabalho foi incluído no comando apresentado anteriormente e pode ser observado quando o Scrapy detecta que está dentro de uma pasta de projeto:

\pagebreak
\lstset{basicstyle=\scriptsize,
caption={Saída da execução do comando ''scrapy'' dentro da pasta de um projeto},
captionpos=b
}
\begin{lstlisting}
Scrapy 0.11 - project: scrapytest

Usage:
  scrapy <command> [options] [args]

Available commands:
  crawl         Start crawling from a spider or URL
  deploy        Deploy project in Scrapyd server
  fetch         Fetch a URL using the Scrapy downloader
  genspider     Generate new spider using pre-defined templates
  importwpt     Create a spider based on a Website Parse Template (WPT) file
  list          List available spiders
  parse         Parse URL (using its spider) and print the results
  queue         Control execution queue
  runserver     Start Scrapy in server mode
  runspider     Run a self-contained spider (without creating a project)
  settings      Get settings values
  shell         Interactive scraping console
  startproject  Create new project
  version       Print Scrapy version
  view          Open URL in browser, as seen by Scrapy

Use "scrapy <command> -h" to see more info about a command

\end{lstlisting}

O comando \texttt{importwpt} lê um arquivo XML utilizando o formato WPT e cria um \emph{spider} do scrapy a partir dele. Aṕos isso, a execução do comando \texttt{scrapy crawl nome_do_spider_gerado} irá executar o processo de extração de informações conforme descrito no XML que foi importado.

%acho que isso deveria ficar nos resultados...
Como este trabalho ainda está em andamento, algumas funcionalidades, tais como exportação dos dados obtidos, não estão presentes. Como este é um projeto de código aberto, há a possibilidade deste protótipo ser desenvolvido por outras pessoas e mais funcionalidades serem adicionadas com o passar do tempo.

No entanto, no que tange á criação de \emph{spiders}, o projeto está funcionando conforme as especificações do WPT (disponíveis em \cite{wpt}). Como estas especificações ainda estão em desenvolvimento, é possível que em um futuro próximo seja necessária a manutenção do projeto a fim de ser compatível com a última versão do WPT.
% resultados.tex
\chapter{Resultados}
\thispagestyle{fancy}

Como resultado deste trabalho, é apresentado uma ferramenta construida para operar junto ao Scrapy. Esta ferramenta utiliza o código do template em XML/WPT como entrada e produz o respectivo spider na linguagem Python, que utiliza o framework Scrapy como base para cumprir os objetivos de extração de informações estruturadas na Web.

A ferramenta aqui apresentada ainda encontra-se em desenvolvimento. Por ser uma contribuição ao Scrapy, um framework de código aberto, este protótipo poderá ser desenvolvido por outros membros da comunidade. O código-fonte deste protótipo encontra-se num repositório do Github (\texttt{http://github.com/}) disponível em \texttt{http://github.com/herberthamaral/}, no \emph{branch} "crawling-template".

Por não se tratar de uma ferramenta estável, o código originado do presente trabalho ainda não foi incorporado na linha de desenvolvimento principal do Scrapy, mas espera-se que isto seja feito em breve.

Apesar de ser um projeto de código aberto e disponibilizado publicamente, ainda não há contribuições de outros desenvolvedores. Um dos motivos é a proposital falta de divulgação: espera-se ter uma versão que possa ser considerada estável antes de apresentar este projeto formalmente à comunidade.

Um dos aspectos positivos da abordagem adotada é que a ferramenta desenvolvida gera o código de spiders na linguagem de programação Python a partir de código em XML, permitindo que o programador faça altarações no código gerado, caso seja necessário. 

Como este trabalho não contempla, tampouco pretende contemplar, todas as facetas da criação de sistemas de recuperação de informações estruturadas na Web, a possibilidade do programador alterar o código gerado pode vir a ser bem útil em situações mais específicas. Por isso o objetivo deste trabalho é \emph{gerar código} ao invés de \emph{interpretar código} com o intuito de criar sistemas de recuperação de informações.

Pelo fato dos objetivos deste trabalho se basearem na criação de uma ferramenta que facilite a criação de sistemas de recuperção de informação na Web, duas análises podem ser feitas com os resultados: uma quantitativa e outra qualitativa. Quantitativa para mostrar as visíveis economias no processo de escrita/manutenção de código. Qualitativa para mostrar como os resultados obtidos podem levar a uma maior produtividade, menor curva de aprendizado e resultados mais rápidos. Também se faz necessária a demonstração de exemplos de sistemas de recuperação de informações reais criados a partir do uso da abordagem desenvolvida neste trabalho, bem como as dificuldades encontras e trabalhos futuros.

\section{Avaliação quantitativa}

A avaliação de sucesso no cumprimento dos objetivos deste projeto pode ser meramente subjetiva, porém, há alguns critérios quantitativos ou subjetivos que podem ser discutidos, como a economia em linhas de código. 

A avaliação quantitativa leva em conta o número de linhas no formato XML/WPT necessárias para gerar o código em Python com o uso da ferramenta desenvolvida neste trabalho e é mensurada através da comparação do número de linhas de código, em ambas linguagens, necessárias para obter o mesmo resultado. Uma avaliação positiva é dada se o número de linhas de código gerado em Python for superior ao número de linhas utilizadas para escrever o template em XML.

No exemplo mais simples possível, há o template mostrado na Listagem \ref{wpt_exemplo}.

\lstset{language=XML,
basicstyle=\scriptsize,
caption={Exemplo de template em WPT},
captionpos=b
}
\begin{lstlisting}[label=wpt_exemplo]
  <ow:wpt xmlns:ow="http://www.omfica.org/schemas/ow/0.9"
            ow:host="http://example.com">
  <ow:template ow:name="Template Example" ow:url="http://www.example.com/index.php">
      <ow:block ow:tagid="ex1" name="ex1"></ow:block>
  </ow:template> 
  </ow:wpt>
\end{lstlisting}

Com o uso da ferramenta desenvolvida neste trabalho, o template anterior corresponde ao código gerado em Python mostrado na Listagem \ref{spider_python_wpt_exemplo}.

\lstset{language=Python,
basicstyle=\scriptsize,
caption={\emph{Spider} gerado a partir do arquivo de entrada apresentado na listagem \ref{wpt_exemplo}},
captionpos=b
}
\begin{lstlisting}[label=spider_python_wpt_exemplo]
from scrapy.item import Item, Field

class TemplateExampleItem1(Item):
    ex1 = Field()

from scrapy.spider import BaseSpider
from scrapy.contrib.loader import XPathItemLoader

class TemplateExample(BaseSpider):
    name = 'example.com'
    allowed_domains = ['example.com']
    start_urls = [
        'http://www.example.com/index.php',
    ]
    
    def parse(self, response):
        l = XPathItemLoader(item = TemplateExampleItem1(),response=response)
        l.add_xpath('bubble','id("ex1")/text()') 
        i = l.load_item()
        yield i
\end{lstlisting}

O código em Python possui 20 linhas, 16 sem as linhas em branco, ao passo que o template possui apenas 5 linhas de código. Neste exemplo, na pior das hipóteses, a economia de código gira em torno de 2/3 do original. A diferença fica mais evidente em um exemplo mais completo, como o mostrado na listagem \ref{wpt_exemplo_mais_completo}.

\lstset{language=XML,
basicstyle=\scriptsize,
caption={Exemplo mais completo de template em WPT},
captionpos=b
}
\begin{lstlisting}[label=wpt_exemplo_mais_completo]
<ow:wpt xmlns:ow="http://www.omfica.org/schemas/ow/0.9"
ow:host="http://example.com">
<ow:template ow:name="Template Example" ow:url="http://www.example.com/index.php">
  <ow:block ow:tagid="ex1" name="ex1"></ow:block>
  <ow:block ow:xpath="/html/body/p/text()" name="paragraph"></ow:block>
  
  <ow:block ow:xpath="/html/body/ul/li" name="menu">
    <ow:block ow:xpath="//a/text()" name="item" ow:type="repeatable"></ow:block>
    <ow:block ow:xpath="//a[@href]" name="url" ow:type="repeatable"></ow:block>
  </ow:block>
  
  <ow:block ow:xpath="/html/body/div[2]/ul/li" name="paginate">
    <ow:block ow:xpath="//a[@href]" name="url" ow:type="follow"></ow:block> <!-- follow -->
  </ow:block>
</ow:template> 
</ow:wpt>
\end{lstlisting}

Com o auxílio da ferramenta citada anteriormente, o código anterior em XML é utilizado para gerar código em Python mostrado na Listagem \ref{spider_python_wpt_exemplo_mais_completo}.

\lstset{language=Python,
basicstyle=\scriptsize,
caption={\emph{Spider} gerado a partir do arquivo de entrada apresentado na listagem \ref{wpt_exemplo_mais_completo}},
captionpos=b
}
\begin{lstlisting}[label=spider_python_wpt_exemplo_mais_completo]
from scrapy.item import Item, Field

class TemplateExampleItem1(Item):
    ex1 = Field()

class TemplateExampleItem2(Item):
    paragraph = Field()
    
class MenuItem(Item):
    item = Field()
    url = Field()
    
class Paginate(Item):
    url = Field()

from scrapy.spider import BaseSpider
from scrapy.contrib.loader import XPathItemLoader
from scrapy.selector import HtmlXPathSelector
from scrapy.http import Request

class TemplateExample(BaseSpider):
    name = 'example.com'
    allowed_domains = ['example.com']
    start_urls = [
        'http://www.example.com/index.php',
    ]

    def parse(self, response):
        l = XPathItemLoader(item = TemplateExampleItem1(),response=response)
        l.add_xpath('bubble','id("ex1")/text()') 
        i = l.load_item()
        yield i
        
        l = XPathItemLoader(item = TemplateExampleItem2(),response=response)
        l.add_xpath('paragraph','/html/body/p/text()') 
        i = l.load_item()
        yield i
        
        hxs = HtmlXPathSelector(response)
        for item in hxs.select('/html/body/ul/li'):
            l = XPathItemLoader(item = MenuItem(), selector=item)
            l.add_xpath('item','//li/a/text()') 
            l.add_xpath('url','//li/a[@href]') 
            i = l.load_item()
            yield i

        hxs = HtmlXPathSelector(response)
        for item in hxs.select('/html/body/div[2]/ul/li'):
            l = XPathItemLoader(item = Paginate(), selector=item)
            l.add_xpath('url','//li/a[@href]') 
            i = l.load_item()
            yield Request(item=i['url'],callback=self.parse)
            yield i
\end{lstlisting}

Neste segundo exemplo, o arquivo XML possui 15 linhas, ao passo que o código gerado possui cerca de 60 linhas, aproximadamente 4 vezes menos código.

\pagebreak
\section{Avaliação qualitativa}

Em uma comparação qualitativa, pode-se citar alguns critérios:

\begin{enumerate}
	\item Linguagens de propósito geral \emph{versus} linguagens de proposíto específico (ou Domain Specific Languages).
	\item Ganho de produtividade.
	\item Facilidade para o usuário final.
	\item Automação de processos.
\end{enumerate}

Linguagens de propósito geral servem para construção de mais diversos tipos de sistemas, ao passo que linguagens de propósito específico são projetadas para resolver problemas em um domínio específico. Há um \emph{tradeoff} entre as duas abordagens: enquanto as linguagens de propósito geral são mais completas, elas também são mais complexas se comparadas com as linguagens de propósito específico para a mesma categoria de problema.

Quando se deixa de usar uma linguagem de propósito geral (Python) para utilizar outra de propósito específico (XML/WPT) o ganho de produtividade estará em eliminar preocupações não inerentes ao domínio. Com menos recursos e preocupações para se trabalhar, o trabalho do usuário final é facilitado.

Por exemplo, quando se trabalha com uma linguagem de propósito geral (como Python, C/C++ ou Java), o programador dispõe de uma maior flexibilidade, pois o mesmo pode valer de estruturas de controle, repetição, variáveis, funções ou métodos, classes, objetos, interação com o sistema de arquivos e E/S, interação com outros sistemas e tratamento de exceções para desenvolver suas soluções. Além de todos os itens relacionados à linguagem de programação, o programador precisa ter domínio de técnicas de extração de informações de páginas na Web, em que, nos casos mais simples envolve o conhecimento de uma linguagem de marcação (HTML, por exemplo), bibliotecas utilizadas para \emph{parsing} e bibliotecas para comunicação com servidores HTTP (\emph{HyperText Transfer Protocol} - Protocolo de Transferência de HiperTexto). No caso de uma linguagem específica de domínio baseada em XML, o programador precisará ter as mesmas noções de extração de informações de páginas na Web, porém sem a complexidade de lidar com uma linguagem de propósito geral e todas as suas respectivas dependências (bibliotecas), limitando-se apenas a conhecer a sintaxe do XML e as regras inerentes ao WPT.

O uso de XML pode trazer várias vantagens à criação de sistemas de recuperação de informação. A principal é que qualquer linguagem de programação que dê suporte à XML pode criar templates. Como o trabalho de \emph{não} escrever código é sempre menor do que escrever \emph{algum} código, os usuários podem se beneficiar com a criação de sistemas que geram templates automaticamente. Por exemplo, um \emph{plugin} de um navegador Web pode permitir que o usuário selecione as seções de uma determinada página que gostaria que fossem recuperadas automaticamente e então gerar o template para o mesmo. 

Uma outra vantagem do uso de XML como meio para geração destes sistemas é a baixa dependência de uma única linguagem de programação, como a linguagem Python utilizada neste trabalho, uma vez que qualquer outra linguagem com as devidas bibliotecas pode interpretar XML e gerar código para qualquer outra linguagem de programação. Com isso, há como diminuir o nível de acoplamento entre a solução desenvolvida e a linguagem utilizada, de forma que todo o código em XML possa ser reaproveitado para geração de outros sistemas em outras linguagens de programação sem maiores problemas.

\pagebreak
\section{Exemplos reais}

Para fornecer melhores bases de entendimento deste trabalho, a seguir são apresentadas exemplos de casos de uso da ferramenta desenvolvida em conjunto com todo o processo de identificação de elementos na página e escrita do template.

\subsection{Exemplo 2 - Página de editais do Ministério da Cultura}

A página de editais do Ministério da Cultura, que se encontra em \url{http://www.cultura.gov.br/site/categoria/editais-ministerio-da-cultura/}, contém a lista dos últimos editais publicados, com seus respectivos títulos, categorias e informações gerais, como demonstrado na figura \ref{minc}.

\begin{figure} [ht]
	\centering
	\includegraphics[scale=0.8]{minc.png}
	\caption{Seleção dos elementos que compõem uma chamada de notícia de edital}
	\label{minc}
\end{figure}

A seguir, as legendas das respectivas áreas destacadas:

\begin{enumerate}
	\item Título da notícia
	\item Metadados (data e tags)
	\item Mais informações
\end{enumerate}

Analisando o código HTML da seleção, obtemos os seguintes seletores XPATH:

\begin{itemize}
	\item \textbf{Base} - \texttt{/html/body/div[2]/div[1]/ol[1]/li}
	\item \textbf{Título} - \texttt{//h3/a/text()}
	\item \textbf{Metadados} - \texttt{//p/text()}
	\item \textbf{Mais informações} - \texttt{//div/p/text()}
\end{itemize}

Como os itens desejados (chamadas de notícias) se repetem na página, é necessário indicar qual será o padrão de repetição, o que neste caso é indicado pelo seletor \textbf{Base}: todos os itens de lista (\texttt{li}) que se encontram em \texttt{/html/body/div[2]/div[1]/ol[1]}. Os demais seletores derivam de cada item da lista indicada por \textbf{Base}.

Seguindo as especificações do Website Parse Template (WPT), o template necessário para obter a lista de editais com seus respetivos atributos se encontra na listagem \ref{wpt_minc}.

\lstset{language=XML,
basicstyle=\scriptsize,
caption={Template utilizado para recuperar os editais do sítio do Ministério da Cultura},
captionpos=b
}
\begin{lstlisting}[label=wpt_minc]
<ow:wpt xmlns:ow="http://www.omfica.org/schemas/ow/0.9"
 ow:host="http://example.com">
<ow:template ow:name="minc_news" 
 ow:url="http://www.cultura.gov.br/site/categoria/editais-ministerio-da-cultura/">
  <ow:block ow:xpath="/html/body/div[2]/div[1]/ol[1]/li" name="lista_news" ow:type="repeatable">
    <ow:block ow:xpath="//h3/a/text()" name="titulo" ></ow:block>
    <ow:block ow:xpath="//p/text()" name="meta"></ow:block>
    <ow:block ow:xpath="//div/p/text()" name="mais_info"></ow:block>
  </ow:block>
</ow:template> 
</ow:wpt>
\end{lstlisting}

O template anterior reflete nas regras definidas anteriormente, com o bloco \texttt{lista\_news} como seletor dos itens de lista onde ficam as chamadas dos editais e contendo três blocos: o \texttt{titlulo}, o \texttt{meta} e o \texttt{mais\_info}, que representam as informações de cada chamada edital.

Tendo salvo o template anterior num arquivo chamado \texttt{minc.xml}, o comando \texttt{scrapy importwpt minc.xml} exibe a saída mostrada na listagem \ref{python_wpt_minc}.

\pagebreak
\lstset{language=Python,
basicstyle=\scriptsize,
caption={Código gerado pela ferramenta desenvolvida utilizando o arquivo \texttt{minc.xml} da listagem \ref{wpt_minc} como entrada},
captionpos=b
}
\begin{lstlisting}[label=python_wpt_minc]
from scrapy.item import Item, Field

class minc_newsItem1(Item):
    lista_news = Field()
    titulo = Field()
    meta = Field()
    mais_info = Field()


from scrapy.spider import BaseSpider
from scrapy.contrib.loader import XPathItemLoader

class minc_news(BaseSpider):
    name = 'www.cultura.gov.br'
    allowed_domains = ['www.cultura.gov.br']
    start_urls = ['http://www.cultura.gov.br/site/categoria/editais-ministerio-da-cultura/']

    def parse(self, response):
        
        hxs = HtmlXPathSelector(response)
        for item in hxs.select('/html/body/div[2]/div[1]/ol[1]/li'):
            l = XPathItemLoader(item = minc_newsItem2(), selector=item)
            l.add_xpath('titulo','//h3/a/text()')
            l.add_xpath('meta','//p/text()')
            l.add_xpath('mais_info','//div/p/text()')

            i = l.load_item()
            yield i

\end{lstlisting}


O spider demonstrado na listagem \ref{python_wpt_minc} pode ser executado com o comando \texttt{scrapy runspider /caminho/para/minc.py} e produz a saída (devidamente adaptada para uma melhor folhas de papel A4) mostrada na listagem \ref{scrapy_wpt_minc}

\lstset{basicstyle=\scriptsize,
caption={Saída produzida pela execução do scraper da listagem \ref{python_wpt_minc}},
captionpos=b
}
\begin{lstlisting}[label=scrapy_wpt_minc]
scrapy runspider testescrapy/spiders/minc.py
2010-12-02 07:49:31-0200 [scrapy] INFO: Scrapy 0.11.0 started (bot: testescrapy)
2010-12-02 07:49:31-0200 [scrapy] DEBUG: Enabled extensions: TelnetConsole, SpiderContext, 
WebService, CoreStats, MemoryUsage, CloseSpider
															  
2010-12-02 07:49:31-0200 [scrapy] DEBUG: Enabled scheduler middlewares: DuplicatesFilterMiddleware

2010-12-02 07:49:31-0200 [scrapy] DEBUG: Enabled downloader middlewares: HttpAuthMiddleware, 
DownloadTimeoutMiddleware, UserAgentMiddleware, RetryMiddleware, DefaultHeadersMiddleware, 
RedirectMiddleware, CookiesMiddleware, HttpCompressionMiddleware, DownloaderStats

2010-12-02 07:49:31-0200 [scrapy] DEBUG: Enabled spider middlewares: HttpErrorMiddleware, 
OffsiteMiddleware, RefererMiddleware, UrlLengthMiddleware, DepthMiddleware
																     
2010-12-02 07:49:31-0200 [scrapy] DEBUG: Enabled item pipelines: 
2010-12-02 07:49:31-0200 [scrapy] DEBUG: Telnet console listening on 0.0.0.0:6023
2010-12-02 07:49:31-0200 [scrapy] DEBUG: Web service listening on 0.0.0.0:6080
2010-12-02 07:49:31-0200 [www.cultura.gov.br] INFO: Spider opened
2010-12-02 07:49:32-0200 [www.cultura.gov.br] DEBUG: Crawled (200) 
<GET http://www.cultura.gov.br/site/categoria/editais-ministerio-da-cultura/> (referer: None)
Item recuperado: <titulo=Divulgada lista dos convocados para oficina presencial de Florianopolis> 

Item recuperado: <titulo=Premio Hip Hop> 

Item recuperado: <titulo=Edital Ideias Criativas para 20 de novembro de 2010 - 
						 Dia Nacional da Consciencia Negra> 

Item recuperado: <titulo=Edital do Programa de Apoio a Traducao de Autores Brasileiros> 

Item recuperado: <titulo=Premio Luso-Brasileiro de Dramaturgia Antonio Jose da Silva - 2010> 

Item recuperado: <titulo=Premio Mais Cultura Pontos de Leitura do estado do Ceara> 

Item recuperado: <titulo=Programa de Intercambio e Difusao Cultural - Edital numero 1/2010> 

Item recuperado: <titulo=Edital de Pontos de Leitura de Sao Leopoldo> 

Item recuperado: <titulo=Premio Mais Cultura de Apoio as Bibliotecas Comunitarias do Estado do Ceara> 

Item recuperado: <titulo=Edital 22 Anos da Palmares> 

Item recuperado: <titulo=Premio Mais Cultura de Pontinhos de Cultura de Fortaleza 2010> 

Item recuperado: <titulo=Premio Mais Cultura de Pontos de Leitura de Fortaleza> 

2010-12-02 07:49:32-0200 [www.cultura.gov.br] INFO: Closing spider (finished)
2010-12-02 07:49:32-0200 [www.cultura.gov.br] INFO: Spider closed (finished)
\end{lstlisting}


\subsection{Exemplo 2 - Obtendo as últimas perguntas no StackOverflow}

O StackOverflow é um sítio gratuito de perguntas e respostas sobre programação \cite{stackoverflow}. O objetivo deste exemplo é recuperar as últimas perguntas feitas com alguns de seus respectivos dados: resumo da pergunta, usuário que a fez, número de visualizações, número de votos e número de respostas.

A página em questão (\url{http://stackoverflow.com/about}) contém um índice com últimas perguntas feitas em que cada índice possui a estrutura descrita na Figura \ref{stackoverflow}.

\begin{figure} [ht]
	\centering
	\includegraphics[scale=0.8]{stackoverflow.png}
	\caption{Seleção dos elementos que compõem um item da lista de perguntas do StackOverflow}
	\label{stackoverflow}
\end{figure}

A seguir, as legendas das respectivas áreas destacadas:

\begin{enumerate}
	\item Número de votos
	\item Número de respostas
	\item Número de visualizações
	\item Título da pergunta
	\item Resumo da pergunta
	\item Nome do usuário que fez a pergunta
\end{enumerate}

Analisando o código HTML da seleção, obtemos os seguintes seletores XPATH:

\begin{itemize}
	\item \textbf{Base} - \texttt{//div[@id='questions']/div}
	\item \textbf{Titulo} - \texttt{//div[@class='summary']/h3/a/text()}
	\item \textbf{Resumo} - \texttt{//div[@class='summary']/div[1]/text()}
	\item \textbf{Usuário} - \texttt{//div[@class='summary']/div[3]/div[3]/a/text()}
	\item \textbf{Vizualizações} - \texttt{//div[@class='statscontainer']/div[@class='views']/text()}
	\item \textbf{Votos} - \texttt{//div[1]/div[2]/div[1]/div[@class='votes']/span/strong/text()}
	\item \textbf{Respostas} - \texttt{//div[1]/div[2]/div[2]/strong/text()}
\end{itemize}

O item \textbf{Base} indica a base de repetição da lista de perguntas da página. Dentro do container \textbf{Base} encontra-se os outros elementos que compõe o item a ser recuperado. O WPT equivalente às seleções anteriores é descrito na listagem \ref{wpt_stackoverflow}.

\lstset{language=XML,
basicstyle=\scriptsize,
caption={Template utilizado para recuperar as últimas perguntas do sítio do StackOverflow},
captionpos=b
}
\begin{lstlisting}[label=wpt_stackoverflow]
<ow:wpt xmlns:ow="http://www.omfica.org/schemas/ow/0.9"
    ow:host="http://stackoverflow.com">
      
    <ow:template ow:name="stackoverflow" 
        ow:url="http://stackoverflow.com/questions">
       
        <ow:block ow:xpath="//div[@id='questions']/div" name="questions" ow:type="repeatable"> 
            <ow:block ow:xpath="//div[@class='summary']/h3/a/text()" name="tiulo" >
            </ow:block>
            
            <ow:block ow:xpath="//div[@class='summary']/div[1]/text()" name="resumo"> 
            </ow:block>
            
            <ow:block ow:xpath="//div[@class='summary']/div[3]/div[3]/a/text()" name="usuario" >
            </ow:block>

            <ow:block ow:xpath="//div[@class='statscontainer']/div[@class='views']/text()" 
                name="views" ></ow:block>

            <ow:block ow:xpath="//div[1]/div[2]/div[1]/div[@class='votes']/span/strong/text()" 
                name="votes" >
            </ow:block>

            <ow:block ow:xpath="//div[1]/div[2]/div[2]/strong/text()" name="answers" >
            </ow:block>
        </ow:block>
              
     </ow:template> 

</ow:wpt>
\end{lstlisting}

Tendo salvo o conteúdo descrito na listagem \ref{wpt_stackoverflow} no arquivo \texttt{stackoverflow.xml}, o comando para gerar o respectivo código em Python é \texttt{scrapy importwpt stackoverflow.xml}. O código gerado é mostrado na listagem \ref{py_stackoverflow}.

\lstset{language=XML,
basicstyle=\scriptsize,
caption={Template utilizado para recuperar os editais do sítio do Ministério da Cultura},
captionpos=b
}
\begin{lstlisting}[label=py_stackoverflow]

from scrapy.item import Item, Field

class stackoverflowItem1(Item):
    questions = Field()
    tiulo = Field()
    resumo = Field()
    usuario = Field()
    views = Field()
    votes = Field()
    answers = Field()


from scrapy.spider import BaseSpider
from scrapy.contrib.loader import XPathItemLoader
from scrapy.selector import HtmlXPathSelector
from scrapy.http import Request

class stackoverflow(BaseSpider):
    name = 'stackoverflow.com'
    allowed_domains = ['stackoverflow.com']
    start_urls = ['http://stackoverflow.com/questions']

    def parse(self, response):
        
        hxs = HtmlXPathSelector(response)
        for item in hxs.select('//div[@id="questions"]/div'):
            l = XPathItemLoader(item = stackoverflowItem1(), selector=item)
            l.add_xpath('tiulo','//div[@class="summary"]/h3/a/text()')
            l.add_xpath('resumo','//div[@class="summary"]/div[1]/text()')
            l.add_xpath('usuario','//div[@class="summary"]/div[3]/div[3]/a/text()')
            l.add_xpath('views','//div[@class="statscontainer"]/div[@class="views"]/text()')
            l.add_xpath('votes','//div[1]/div[2]/div[1]/div[@class="votes"]/span/strong/text()')
            l.add_xpath('answers','//div[1]/div[2]/div[2]/strong/text()')

            i = l.load_item()
            yield i
\end{lstlisting}

A Listagem \ref{out_stack} mostra uma saída adaptada da execução do spider descrito na Listagem \ref{py_stackoverflow}.

\begin{lstlisting}[label=out_stack]
2010-12-04 21:04:17-0200 [scrapy] INFO: Scrapy 0.11.0 started (bot: testescrapy)
2010-12-04 21:04:17-0200 [scrapy] DEBUG: Enabled extensions: TelnetConsole, 
SpiderContext, WebService, CoreStats, MemoryUsage, CloseSpider

2010-12-04 21:04:17-0200 [scrapy] DEBUG: Enabled scheduler middlewares: DuplicatesFilterMiddleware
2010-12-04 21:04:17-0200 [scrapy] DEBUG: Enabled downloader middlewares: HttpAuthMiddleware, 
  DownloadTimeoutMiddleware, UserAgentMiddleware, RetryMiddleware, DefaultHeadersMiddleware, 
  RedirectMiddleware, CookiesMiddleware, HttpCompressionMiddleware, DownloaderStats
  
2010-12-04 21:04:17-0200 [scrapy] DEBUG: Enabled spider middlewares: HttpErrorMiddleware, 
OffsiteMiddleware, RefererMiddleware, UrlLengthMiddleware, DepthMiddleware
2010-12-04 21:04:17-0200 [scrapy] DEBUG: Enabled item pipelines: 
2010-12-04 21:04:17-0200 [scrapy] DEBUG: Telnet console listening on 0.0.0.0:6023
2010-12-04 21:04:17-0200 [scrapy] DEBUG: Web service listening on 0.0.0.0:6080
2010-12-04 21:04:17-0200 [stackoverflow.com] INFO: Spider opened
2010-12-04 21:04:23-0200 [stackoverflow.com] DEBUG: Crawled (200) 
    <GET http://stackoverflow.com/questions> (referer: None)
    
Item recuperado <titulo=C# - sorting a list inside of a struct,usuario=Tyler>
Item recuperado <titulo=JavaScript - Smooth Movement / Resizing,usuario=jSherz>
Item recuperado <titulo=C++ Linker Errors,usuario=Yelnats>
Item recuperado <titulo=How can I access my trackpad in C#?,usuario=Serg>
Item recuperado <titulo=(English, Perl, Python, Ruby) comparison on a code 
  fragment-by-fragment basis?,usuario=blunders>
Item recuperado <titulo=MySQL 5.1 Memory Table ,usuario=James>
Item recuperado <titulo=How to access the file which is in Production Server 
  from a class in a Jar,usuario=kumar>

Item recuperado <titulo=Does anyone know where I can find some good WebGl 
  documentation?,usuario=Catalin Dumitru>
  
Item recuperado <titulo=Scaml syntax highlighting on Emacs,usuario=alp247>
Item recuperado <titulo=Cx_freeze - How can I Install the shared libraries 
  to /usr/lib,usuario=MetaDark>
  
Item recuperado <titulo=PHP random string generator,usuario=ssen>
Item recuperado <titulo=How do I use Watir::Waiter::wait_until to force
  Chrome to wait?,usuario=karim79>
Item recuperado <titulo=React differently if second time user visits domain,
  usuario=Mike Grace>
  
Item recuperado <titulo=return flat text from .net webservice ? ,usuario=user391179>
Item recuperado <titulo=General Questions about Entity Framework 
  vs. Enterprise Library & a few others,usuario=icode.cs>
  
2010-12-04 21:04:23-0200 [stackoverflow.com] INFO: Closing spider (finished)
2010-12-04 21:04:23-0200 [stackoverflow.com] INFO: Spider closed (finished)
\end{lstlisting}

%----------------%<---------------------



\pagebreak
\chapter{Considerações Finais}
\thispagestyle{fancy}

\section{Dificuldades encontradas}

Apesar da abundante documentação e mesmo não possuindo uma má qualidade de código ou baixa cobertura de testes, uma das principais dificuldades foi aderir à arquitetura e estilo de codificação do Scrapy para que a padronização entre componentes fosse mantida.

O framework em si possui um tamanho considerável (759 arquivos em 13/11/2010 \footnote{Informação obtida através do comando \texttt{find | wc -l} }), portanto, leva-se um tempo para acostumar-se com a estrutura de diretórios, arquivos e padrões de nome.

Não foi possível encontrar referências acadêmicas do Scrapy, um dos motivos pelos quais somente a respectiva documentação oficial foi utilizada.

O formato WPT possui apenas informações sobre a disposição dos elementos em uma página e seus respectivos significados (ontologia). Tal fato leva ao incoveniente de haver diferentes \emph{schemas} XML em um mesmo arquivo para determinar outras informações, como nome dos variáveis do código gerado e formato de persistência dos dados obtidos.

Ainda não é possível o uso de WPT para determinar o verbo HTTP a ser utilizado. Atualmente, todas as requisições feitas com os spiders gerados a partir do WPT são feitas através do HTTP GET, não sendo possível utilizar outros verbos HTTP como POST, PUT, DELETE e PATCH. 

As regras de validação formam outra dificuldade relacionada ao WPT: verificação de validade de URLs, caminhos XPATH, validação de templates, blocos e nomes de elementos. Como a validação ocorre em vários elementos e o XML é um formato hierárquico, validações recursivas precisaram ser feitas, e, dependendo do nível de recursão, tratamentos especiais teriam que ser dados. Como um exemplo, considere o trecho de código da listagem \ref{problemas_wpt_blocos1}:

\lstset{language=XML,
basicstyle=\scriptsize,
caption={Hierarquia de blocos planos e de um nível},
captionpos=b
}
\begin{lstlisting}[label=problemas_wpt_blocos1]
 ...
  <ow:block ow:xpath="/bloco1" name="raiz">
    <ow:block ow:xpath="//caminho/para/um" name="raiz_um"></ow:block>
    <ow:block ow:xpath="//caminho/para/dois" ow:type="repeatable" name="raiz_dois"></ow:block>
  </ow:block>
 
  <ow:block ow:xpath="/bloco2" name="dois"></ow:block>
  ...
\end{lstlisting}

O bloco com nome \texttt{dois} é um bloco de nível único e o bloco com nome \texttt{raiz} é um bloco hierárquico com um nível. Os blocos de nível único são fáceis de serem interpretados pois representam apenas um item na página a ser obtido. Já os blocos com um nível formam itens compostos por um ou mais elementos de nível único. Blocos com o atributo \emph{ow:type="repeatable"} (como o bloco de nome \texttt{raiz\_dois}) em um contexto que nem todos são repetíveis precisam de um tratamento especial. Blocos de nível único repetíveis são tratados de maneira similar.

O principal problema acontece quando há um bloco com um nível maior que um, como o mostrado na listagem \ref{problemas_wpt_blocos2}. Este tipo de caso ainda não é tratado pela ferramenta, por ainda ser complexo de lidar. De fato, ainda não há uma solução definitiva para blocos com mais de um nível, pois todas as regras anteriores, incluindo verificação de nomes e verificação de blocos com repetições, devem ser aplicadas.

\lstset{language=XML,
basicstyle=\scriptsize,
caption={Hierarquia de blocos de dois níveis},
captionpos=b
}
\begin{lstlisting}[label=problemas_wpt_blocos2]
 ...
  <ow:block ow:xpath="/bloco_raiz" name="raiz">

    <ow:block ow:xpath="//caminho/para/um" name="um">
    	<ow:block ow:xpath="//caminho/para/um_pt_um" name="um_pt_um"></ow:block>
    	<ow:block ow:xpath="//caminho/para/um_pt_dois" name="um_pt_dois"></ow:block>
    </ow:block>
    
    <ow:block ow:xpath="//caminho/para/dois" name="dois"></ow:block>
    <ow:block ow:xpath="//caminho/para/tres" name="tres"></ow:block>
    
  </ow:block>
  ...
\end{lstlisting}

Questões como "quantos níveis poderão ser suportados?", "como os blocos do nível N deverão ser tratados?","qual a profundidade máxima de níveis que a ferramenta poderá suportar?", "como blocos repetíveis serão tratados em cada nível", "como ficará a questão da navegação em outras páginas a partir de links para serem seguidos forem detectados em blocos de níveis mais inferiores?" ainda permanecem sem resposta, devido à natureza e complexidade de achar a resposta das mesmas.

Há ainda a preocupação com a qualidade e legibilidade do código gerados, uma vez que a ferramenta desenvolvida não visa substituir o programador, mas sim auxilia-lo. A ferramenta precisa gerar código devidamente organizado e com alguns comentários para melhor orientação. A indentação obrigatória da linguagem Python é uma característica que ajuda neste quesito, porém não deixa de causar problemas com relação a indentação na geração de código.

A biblioteca padrão para testes disponíveis no Python 2.6 (utilizado neste trabalho), a \texttt{unittest}, não possui um método de comparação de strings linha-a-linha, que é uma funcionalidade útil quando se trata testes de métodos que geram código. Isso significa que, se um texto precisa ser comparado com outro em um teste unitário, os dois serão comparados em sua totalidade, não linha-a-linha, o que dificulta os testes dos métodos responsáveis por geração de código.

Esta dificuldade torna os testes de geração de código mais trabalhosos, pois é necessário observar cada caractere com cautela quando os dois textos, o esperado e o obtido, falham no teste unitário.

Outra dificuldade encontrada nos testes unitários foi o teste de comandos, devido ao tempo que os testes levam para serem executados. A suíte de testes do Scrapy levam aproximadamente 30 segundos para serem executados\footnote{Os testes são executados com o comando \texttt{bin/runtests.sh scrapy.tests} dentro do diretório raiz do código fonte do Scrapy }, o que atrasa o desenvolvimento, já que a cada modificação no código, os testes precisam ser executados.
\vfill

\pagebreak
\section{Trabalhos futuros}

O atual sistema ainda possui algumas limitações, como a falta de persistência dos dados obtidos. No atual cenário, é necessário que o programador altere o código fonte gerado para que os itens gerados pelo Scrapy durante a recuperação de informações sejam persistidos em disco.

O presente trabalho conseguiu atingir o objetivo de facilitar a criação e manutenção de sistemas de recuperação estruturada na Web, porém é possível facilitar ainda mais este trabalho. Plugins para navegadores Web podem ser criados de forma que os usuários possam criar os Website Parse Templates (WPT) através de uma interface visual.

Como o significado dos dados obtidos (metadados) não fazem parte deste trabalho, a seção \texttt{<ow:ontology>} do WPT é deliberadamente ignorada. Em trabalhos futuros, é provável que a mesma seja utilizada e então o protótipo aqui criado terá de ser modificado. 

\emph{Plugins} como o Firebug \cite{firebug} para o navegador Mozilla Firefox permitem inspecionar a estrutura do conteúdo de sítios na Wieb. Com ele, é possível selecionar elementos, mudar estilos e conteúdo, analisar o tráfego utilizado pelo acesso ao sítio e \emph{debugar} JavaScript. Usando o Firebug como plataforma, é possível desenvolver um outro complemento que facilite a criação de Htmltemplates, o que poderia facilitar ainda mais o trabalho de criação de sistemas de recuperção de informações na Web.

% apendice.tex
\appendix
\chapter{Arquivos fonte}

\lstset{language=Python,
basicstyle=\scriptsize,
caption={Arquivo fonte do comando gerado para o Scrapy, o \texttt{importwpt.py}}
}
\begin{lstlisting}

from scrapy.command import ScrapyCommand
from scrapy.utils.misc import load_object
from scrapy.conf import settings
from scrapy.exceptions import UsageError
from scrapy import log
from lxml import objectify
from lxml.etree import XMLSyntaxError
import re
import sys
import string

XMLNS = '{http://www.omfica.org/schemas/ow/0.9}' 

spider_template = """from scrapy.spider import BaseSpider
from scrapy.contrib.loader import XPathItemLoader

class ${spider_class}(BaseSpider):
    name = ${name}
    allowed_domains = [${name}]
    start_urls = [${start_urls}]

    def parse(self, response):
        ${item_load}
"""

item_template = """
class ${item}(Item):
    ${fields}
"""

field_template="    ${field}=Field(${default})"

item_load_template = """l = XPathItemLoader(item = TemplateExampleItem1(),response=response)
        l.add_xpath('bubble','id("ex1")/text()') 
        i = l.load_item()"""


wpt_url = "http://www.w3.org/Submission/WPT/"

log.start()

class Command(ScrapyCommand):
    """
    Do a import of a WPT file 
    """
    requires_project = True

    def syntax(self):
        return "<file.xml>"

    def short_desc(self):
        return "Create a spider based on a Website Parse Template (WPT) file"

    def add_options(self, parser):
        ScrapyCommand.add_options(self, parser)

    def run(self, args, opts):

        if len(args) != 1:
            raise UsageError()

        filename = args[0]

        try:
            template = open(filename).read()

            root = objectify.fromstring(template)
            if not self._check_if_wpt_file_has_valid_url(root):
                log.msg("ERROR: attribute 'host' of 'wpt' tag does not have a valid URL",log.ERROR)

            if not self._check_if_wpt_has_at_least_one_template_with_one_block(xml):
                log.msg("ERROR: The WPT file must have at least one template tag\
                        with one block tag. See %s for more info" % wpt_url,log.ERROR);
    
            if not self._check_if_every_template_has_at_least_one_block(xml):
                log.msg("ERROR: Every template tag must have at least one block tag",log.ERROR)
    
            if not self._check_if_every_block_has_at_least_one_html_element_reference(xml):
                log.msg("ERROR: Every block tag must have at least one \
                        specific HTML tag reference (tagid, pattern or xpath).\
                        See %s for more info" % wpt_url,log.ERROR)

            if not self._check_if_every_template_has_a_unique_name(xml):
                log.msg("ERROR: Each template's name must be unique")
            
            if not self._check_if_url_section_is_valid_if_templates_has_no_urls(xml):
                log.msg("ERROR: Every template which does not have an 'ow:url'\
                        attribute must declare it in urls section. \
                        See %s for more info" % wpt_url,log.ERROR )

        except XMLSyntaxError,e:
            log.msg("ERROR: There is a markup error in %s" %filename,log.ERROR)
        except:
            log.msg("ERROR: File not found: %s" % filename,log.ERROR)
       
    def _check_if_wpt_file_has_valid_url(self,xml):
        try:
            url = xml.attrib[XMLNS+'host']
            p = re.compile('https?://([-A-Za-z0-9+&@#/%?=~_()|!:,.;]*[-A-Za-z0-9+&@#/%=~_()|])')
            domains = p.findall(url)
            
            if not domains: 
                return False
    
     
            domain = domains[0]

            self.domain = domain 

            if domain.__contains__('/'):
                domain = domain.split('/')[0]
     
            site_name = domain.split('.')[0]
     
            if site_name == 'www':
                site_name = domain.split('.')[1]

            self.site_name = site_name

        except:
            return False
    
        return True

    def _check_if_every_block_has_at_least_one_html_element_reference(self,xml):
        for t in xml.template:
            if not _check_if_every_block_has_at_least_one_html_element_reference(t.block):
                return False

        return True
   
    def _check_if_block_has_at_least_one_html_element_reference(self,block):
        if hasattr(block,'block'): 
            if not self._check_if_block_has_at_least_one_html_element_reference(block.block):
                return False

        for b in block:
            has_xpath = b.attrib.__contains__(XMLNS+'xpath')
            has_tagid = b.attrib.__contains__(XMLNS+'tagid')
            has_pattern = b.attrib.__contains__(XMLNS+'pattern')
            if not (has_xpath or has_tagid or has_pattern):
                return False

        return True
 
    def _check_if_every_template_has_at_least_one_block(self,xml):
        try:
            for i in range(0,xml.template.__len__()):
                if not self._template_has_block(xml,i):
                    return False
        except AttributeError:
            return False
        return True

    def _check_if_wpt_has_at_least_one_template_with_one_block(self,xml):
        return self._template_has_block(xml,0)

    def _template_has_block(self,xml,template_id):
        try:
            return hasattr(xml.template[template_id],'block')
        except AttributeError:
            return False
   
    def _check_if_every_template_has_a_unique_name(self,xml):
        template_names = []
        for t in xml.template:
            if t.attrib[XMLNS+'name'] in template_names:
                return False
            template_names.append(t.attrib[XMLNS+'name'])

        return True
    def _check_if_url_section_is_valid_if_templates_has_no_urls(self,xml):
        for t in xml.template:
            if t.attrib.__contains__(XMLNS+'url'):
                continue

            name = t.attrib[XMLNS+'name']
            urls = [u for u in xml.urls if u.attrib[XMLNS+'template']==name]

            if  len(urls)==0:
                return False

        return True

    def generate_spider_from_wpt(self,xml):
        oxml = objectify.fromstring(xml)

        item = self.get_items_from_wpt(oxml)
        t = string.Template(spider_template)
        spider_class = oxml.template.attrib[XMLNS+'name'].replace(' ','')
        name = "'"+self.domain+"'"
        start_urls = "'"+oxml.template.attrib[XMLNS+'url']+"'"
        item_load = self.get_items_load_from_wpt(oxml) 

        py = t.substitute(spider_class=spider_class,name=name,start_urls=start_urls,item_load=item_load)
        
        py = item+"\n"+py
        return py
    
    def get_items_from_wpt(self,xml):
        items = "\nfrom scrapy.item import Item, Field\n"
        blocks = xml.template.block
        class_counter = 1
        item_class_prefix = xml.template.attrib[XMLNS+'name'].replace(' ','')

        for b in blocks:
            class_name = item_class_prefix+"Item"+str(class_counter) 
            class_counter += 1
            fields = "%s = Field()" % b.attrib['name']

            t = string.Template(item_template) 
            items += t.substitute(item=class_name,fields=fields) 
        return items

    def get_items_load_from_wpt(self,xml):
        blocks = xml.template.block
        class_counter = 1
        item_class_prefix = xml.template.attrib[XMLNS+'name'].replace(' ','')
        item_load = ""
        for b in blocks:
            class_name = item_class_prefix+"Item"+str(class_counter)
            class_counter += 1
            t = string.Template(item_load_template)

            if b.attrib.__contains__(XMLNS+'tagid'):
                xpath = 'id("%s")/text()' % b.attrib[XMLNS+'tagid']
            
            field_name = b.attrib['name']

            item_load += t.substitute(class_name=class_name,field_name=field_name,xpath=xpath)

        return item_load
\end{lstlisting}

%-------------------------------------%<-------------------------------

\pagebreak

\lstset{language=Python,
basicstyle=\scriptsize,
caption={Arquivo fonte dos testes automatizados para o comando gerado para o Scrapy. Arquivo \texttt{test\_commands.py}}
}
\begin{lstlisting}

from __future__ import with_statement

import sys
import os
import subprocess
from os.path import exists, join, dirname, abspath
from shutil import rmtree
from tempfile import mkdtemp

from twisted.trial import unittest

import scrapy


class ProjectTest(unittest.TestCase):
    project_name = 'testproject'

    def setUp(self):
        self.temp_path = mkdtemp()
        self.cwd = self.temp_path
        self.proj_path = join(self.temp_path, self.project_name)
        self.proj_mod_path = join(self.proj_path, self.project_name)
        self.env = os.environ.copy()
        self.env['PYTHONPATH'] = dirname(scrapy.__path__[0])

    def tearDown(self):
        rmtree(self.temp_path)

    def call(self, *new_args, **kwargs):
        out = os.tmpfile()
        args = (sys.executable, '-m', 'scrapy.cmdline') + new_args
        return subprocess.call(args, stdout=out, stderr=out, cwd=self.cwd, \
            env=self.env, **kwargs)

    def proc(self, *new_args, **kwargs):
        args = (sys.executable, '-m', 'scrapy.cmdline') + new_args
        return subprocess.Popen(args, stdout=subprocess.PIPE, stderr=subprocess.PIPE, \
            cwd=self.cwd, env=self.env, **kwargs)


class StartprojectTest(ProjectTest):

    def test_startproject(self):
        self.assertEqual(0, self.call('startproject', self.project_name))

        assert exists(join(self.proj_path, 'scrapy.cfg'))
        assert exists(join(self.proj_path, 'testproject'))
        assert exists(join(self.proj_mod_path, '__init__.py'))
        assert exists(join(self.proj_mod_path, 'items.py'))
        assert exists(join(self.proj_mod_path, 'pipelines.py'))
        assert exists(join(self.proj_mod_path, 'settings.py'))
        assert exists(join(self.proj_mod_path, 'spiders', '__init__.py'))

        self.assertEqual(1, self.call('startproject', self.project_name))
        self.assertEqual(1, self.call('startproject', 'wrong---project---name'))


class CommandTest(ProjectTest):

    def setUp(self):
        super(CommandTest, self).setUp()
        self.call('startproject', self.project_name)
        self.cwd = join(self.temp_path, self.project_name)
        self.env['SCRAPY_SETTINGS_MODULE'] = '%s.settings' % self.project_name


class GenspiderCommandTest(CommandTest):

    def test_arguments(self):
        # only pass one argument. spider script shouldn't be created
        self.assertEqual(2, self.call('genspider', 'test_name'))
        assert not exists(join(self.proj_mod_path, 'spiders', 'test_name.py'))
        # pass two arguments <name> <domain>. spider script should be created
        self.assertEqual(0, self.call('genspider', 'test_name', 'test.com'))
        assert exists(join(self.proj_mod_path, 'spiders', 'test_name.py'))

    def test_template(self, tplname='crawl'):
        args = ['--template=%s' % tplname] if tplname else []
        spname = 'test_spider'
        p = self.proc('genspider', spname, 'test.com', *args)
        out = p.stdout.read()
        self.assert_("Created spider %r using template %r in module" % (spname, tplname) in out)
        self.assert_(exists(join(self.proj_mod_path, 'spiders', 'test_spider.py')))
        p = self.proc('genspider', spname, 'test.com', *args)
        out = p.stdout.read()
        self.assert_("Spider %r already exists in module" % spname in out)

    def test_template_basic(self):
        self.test_template('basic')

    def test_template_csvfeed(self):
        self.test_template('csvfeed')

    def test_template_xmlfeed(self):
        self.test_template('xmlfeed')

    def test_list(self):
        self.assertEqual(0, self.call('genspider', '--list'))

    def test_dump(self):
        self.assertEqual(0, self.call('genspider', '--dump=basic'))
        self.assertEqual(0, self.call('genspider', '-d', 'basic'))


class MiscCommandsTest(CommandTest):

    def test_crawl(self):
        self.assertEqual(0, self.call('crawl'))

    def test_list(self):
        self.assertEqual(0, self.call('list'))

class RunSpiderCommandTest(CommandTest):

    def test_runspider(self):
        tmpdir = self.mktemp()
        os.mkdir(tmpdir)
        fname = abspath(join(tmpdir, 'myspider.py'))
        with open(fname, 'w') as f:
            f.write("""
from scrapy import log
from scrapy.spider import BaseSpider

class MySpider(BaseSpider):
    name = 'myspider'

    def start_requests(self):
        self.log("It Works!")
        return []
""")
        p = self.proc('runspider', fname)
        log = p.stderr.read()
        self.assert_("[myspider] DEBUG: It Works!" in log)
        self.assert_("[myspider] INFO: Spider opened" in log)
        self.assert_("[myspider] INFO: Closing spider (finished)" in log)
        self.assert_("[myspider] INFO: Spider closed (finished)" in log)

    def test_runspider_no_spider_found(self):
        tmpdir = self.mktemp()
        os.mkdir(tmpdir)
        fname = abspath(join(tmpdir, 'myspider.py'))
        with open(fname, 'w') as f:
            f.write("""
from scrapy import log
from scrapy.spider import BaseSpider
""")
        p = self.proc('runspider', fname)
        log = p.stderr.read()
        self.assert_("No spider found in file" in log)

    def test_runspider_file_not_found(self):
        p = self.proc('runspider', 'some_non_existent_file')
        log = p.stderr.read()
        self.assert_("File not found: some_non_existent_file" in log)

    def test_runspider_unable_to_load(self):
        tmpdir = self.mktemp()
        os.mkdir(tmpdir)
        fname = abspath(join(tmpdir, 'myspider.txt'))
        with open(fname, 'w') as f:
            f.write("")
        p = self.proc('runspider', fname)
        log = p.stderr.read()
        self.assert_("Unable to load" in log)


class ImportWptCommandTest(CommandTest):

    def test_import_file_not_found(self):
        p = self.proc("importwpt","some_non_existent_file")
        log = p.stderr.read()
        self.assert_("ERROR: File not found: some_non_existent_file" in log,'Log contents: '+log)

    def test_import_existent_file_invalid_markup(self):
        tmpdir = self.mktemp()
        os.mkdir(tmpdir)
        fname = abspath(join(tmpdir,"mytemplate.xml"))
        open(fname,"w").write("")
        p = self.proc("importwpt",fname)
        log = p.stderr.read()

        self.assert_(("ERROR: There is a markup error in %s" % fname) in log,'Log contents: '+log)

    def _test_import_valid_markup_with_no_template_tag(self):
        tmpdir = self.mktemp()
        os.mkdir(tmpdir)
        fname = abspath(join(tmpdir,"mytemplate.xml"))
        
        xml = '<?xml version="1.0" encoding="UTF-8"?>\
                     <ow:wpt xmlns:ow="http://www.omfica.org/schemas/ow/0.9" \
                                ow:host="http://www.example.com"> \
                     </ow:wpt>'
        
        with open(fname,"w") as f:
            f.write(xml)
            p = self.proc('importwpt',fname)
            log = p.stderr.read()

        self.assert_("ERROR: The WPT file must have at least one template tag\
                        with one block tag." in log,"log contents %s" % log)

    def test_validate_business_rules(self):
        from lxml import objectify
        from scrapy.commands.importwpt import Command

        cmd = Command()
        
        "_check_if_wpt_file_has_valid_url"
        xml = '<ow:wpt xmlns:ow="http://www.omfica.org/schemas/ow/0.9"></ow:wpt>'
        
        oxml = objectify.fromstring(xml)
        check = cmd._check_if_wpt_file_has_valid_url(oxml)
        self.assert_(not check)
 
        xml = '<ow:wpt xmlns:ow="http://www.omfica.org/schemas/ow/0.9" \
                ow:host="http://example.com"></ow:wpt>'

        oxml = objectify.fromstring(xml)
        check = cmd._check_if_wpt_file_has_valid_url(oxml)
        self.assert_(check)

        "_check_if_wpt_has_at_least_one_template_with_one_block"
        
        check = cmd._check_if_wpt_has_at_least_one_template_with_one_block(oxml)
        self.assert_(not check)
        xml = '<ow:wpt xmlns:ow="http://www.omfica.org/schemas/ow/0.9" \
                ow:host="http://example.com">\
                <ow:template ow:name="Template Example" ow:url="http://www.example.com/index.php"> \
                    <ow:block></ow:block> \
                </ow:template> \
                </ow:wpt>'
        oxml = objectify.fromstring(xml)
        check = cmd._template_has_block(oxml,0) 
        self.assert_(check)

        check = cmd._check_if_wpt_has_at_least_one_template_with_one_block(oxml)
        self.assert_(check)
        
        "_check_if_block_has_at_least_one_html_element_reference"
        check = cmd._check_if_block_has_at_least_one_html_element_reference(oxml.template.block)
        self.assert_(not check)

        xml = '<ow:wpt xmlns:ow="http://www.omfica.org/schemas/ow/0.9" \
                ow:host="http://example.com">\
                <ow:template ow:name="Template Example" ow:url="http://www.example.com/index.php"> \
                    <ow:block ow:tagid="ex1"></ow:block> \
                </ow:template> \
                </ow:wpt>'

        oxml = objectify.fromstring(xml)
        check = cmd._check_if_block_has_at_least_one_html_element_reference(oxml.template.block) 
        self.assert_(check)

        xml = '<ow:wpt xmlns:ow="http://www.omfica.org/schemas/ow/0.9" \
                 ow:host="http://example.com">\
                 <ow:template ow:name="Template Example" ow:url="http://www.example.com/index.php"> \
                     <ow:block ow:tagid="ex1"> \
                        <ow:block></ow:block> \
                     </ow:block> \
                 </ow:template> \
                 </ow:wpt>'
        oxml = objectify.fromstring(xml)
        check = cmd._check_if_block_has_at_least_one_html_element_reference(oxml.template.block)

        self.assert_(not check)
        xml = '<ow:wpt xmlns:ow="http://www.omfica.org/schemas/ow/0.9" \
                 ow:host="http://example.com">\
                 <ow:template ow:name="Template Example" ow:url="http://www.example.com/index.php"> \
                     <ow:block ow:tagid="ex1"> \
                         <ow:block ow:xpath="/html/body/div/"></ow:block> \
                     </ow:block> \
                 </ow:template> \
                 </ow:wpt>'
        oxml = objectify.fromstring(xml)
        check = cmd._check_if_block_has_at_least_one_html_element_reference(oxml.template.block)
        self.assert_(check)


        "_check_if_every_template_has_a_unique_name"
        check = cmd._check_if_every_template_has_a_unique_name(oxml)
        self.assert_(check)
        xml = '<ow:wpt xmlns:ow="http://www.omfica.org/schemas/ow/0.9" \
                 ow:host="http://example.com">\
                 <ow:template ow:name="Template Example" ow:url="http://www.example.com/index.php"> \
                     <ow:block ow:tagid="ex1"> \
                         <ow:block ow:xpath="/html/body/div/"></ow:block> \
                     </ow:block> \
                 </ow:template> \
                 <ow:template ow:name="Template Example" ow:url="http://www.example.com/index.php"> \
                     <ow:block ow:tagid="ex1"> \
                         <ow:block ow:xpath="/html/body/div/"></ow:block> \
                     </ow:block> \
                 </ow:template> \
                 </ow:wpt>'

        oxml = objectify.fromstring(xml)
        check = cmd._check_if_every_template_has_a_unique_name(oxml)
        self.assert_(not check)

        xml = '<ow:wpt xmlns:ow="http://www.omfica.org/schemas/ow/0.9" \
                 ow:host="http://example.com">\
                 <ow:template ow:name="Template Example" ow:url="http://www.example.com/index.php"> \
                     <ow:block ow:tagid="ex1"> \
                         <ow:block ow:xpath="/html/body/div/"></ow:block> \
                     </ow:block> \
                 </ow:template> \
                 <ow:template ow:name="Template Example 2" ow:url="http://www.example.com/index.php"> \
                     <ow:block ow:tagid="ex1"> \
                         <ow:block ow:xpath="/html/body/div/"></ow:block> \
                     </ow:block> \
                 </ow:template> \
                 </ow:wpt>'

        oxml = objectify.fromstring(xml)
        check = cmd._check_if_every_template_has_a_unique_name(oxml)
        self.assert_(check)

        "_check_if_url_section_is_valid_if_templates_has_no_urls"
        check = cmd._check_if_url_section_is_valid_if_templates_has_no_urls(oxml)
        self.assert_(check)

        xml = '<ow:wpt xmlns:ow="http://www.omfica.org/schemas/ow/0.9" \
                 ow:host="http://example.com">\
                 <ow:template ow:name="Template Example"> \
                     <ow:block ow:tagid="ex1"> \
                         <ow:block ow:xpath="/html/body/div/"></ow:block> \
                     </ow:block> \
                 </ow:template> \
                 <ow:template ow:name="Template Example 2" ow:url="http://www.example.com/index2.php"> \
                     <ow:block ow:tagid="ex1"> \
                        <ow:block ow:xpath="/html/body/div/"></ow:block> \
                     </ow:block> \
                 </ow:template> \
                 <ow:urls ow:name="Another Template" ow:template="Another Template"> \
                    <ow:url>http://www.example.com/index.php</ow:url> \
                 </ow:urls> \
                 </ow:wpt>'
        
        oxml = objectify.fromstring(xml)
        check = cmd._check_if_url_section_is_valid_if_templates_has_no_urls(oxml)
        self.assert_(not check)

        xml = '<ow:wpt xmlns:ow="http://www.omfica.org/schemas/ow/0.9" \
                 ow:host="http://example.com">\
                 <ow:template ow:name="Template Example"> \
                     <ow:block ow:tagid="ex1"> \
                         <ow:block ow:xpath="/html/body/div/"></ow:block> \
                     </ow:block> \
                 </ow:template> \
                 <ow:template ow:name="Template Example 2" ow:url="http://www.example.com/index2.php"> \
                     <ow:block ow:tagid="ex1"> \
                         <ow:block ow:xpath="/html/body/div/"></ow:block> \
                     </ow:block> \
                 </ow:template> \
                 <ow:urls ow:name="Template Example" ow:template="Template Example"> \
                    <ow:url>http://www.example.com/index.php</ow:url> \
                 </ow:urls> \
                 </ow:wpt>'

        oxml = objectify.fromstring(xml)
        check = cmd._check_if_url_section_is_valid_if_templates_has_no_urls(oxml)
        self.assert_(check)

    def test_template_generation(self):
        from lxml import objectify
        from scrapy.commands.importwpt import Command

        cmd = Command()

        xml = """<ow:wpt xmlns:ow="http://www.omfica.org/schemas/ow/0.9"
                            ow:host="http://example.com">
                   <ow:template ow:name="Template Example" ow:url="http://www.example.com/index.php">
                      <ow:block ow:tagid="ex1" name="ex1"></ow:block>
                    </ow:template> 
                 </ow:wpt>
             """
        py = """
from scrapy.item import Item, Field

class TemplateExampleItem1(Item):
    ex1 = Field()

from scrapy.spider import BaseSpider
from scrapy.contrib.loader import XPathItemLoader

class TemplateExample(BaseSpider):
    name = 'example.com'
    allowed_domains = ['example.com']
    start_urls = ['http://www.example.com/index.php']

    def parse(self, response):
        l = XPathItemLoader(item = TemplateExampleItem1(),response=response)
        l.add_xpath('bubble','id("ex1")/text()') 
        i = l.load_item()
"""
        self.assert_(cmd._check_if_wpt_file_has_valid_url(objectify.fromstring(xml)))
        spider = cmd.generate_spider_from_wpt(xml)
        self.assertEqual(spider,py)

\end{lstlisting}



\clearpage
\addcontentsline{toc}{chapter}{Bibliografia}
\bibliographystyle{abnt-alf}
\bibliography{main}

\end{document}