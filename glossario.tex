% glossário

\newacronym{wpt}{WPT}{\emph{Website Parse Template}}
\newacronym{www}{WWW}{\emph{World Wide Web}}
\newacronym{dsl}{DSL}{\emph{Domain Specific Language} ou Linguagem específica de domínio}
\newacronym{repl}{REPL}{\emph{Read Eval Print Loop}}
\newacronym{html}{HTML}{\emph{HyperText Markup Language}}
\newacronym{sri}{SRI}{Sistema de Recuperação de informações}
\newacronym{jvm}{JVM}{\emph{Java Virtual Machine}}
\newacronym{cli}{CLI}{\emph{Common Language Infrastructure}Runaway argument?}

\newglossaryentry{webcrawler}
				{name={\emph{Web Crawler}},
				 description={Programa de computador que navega pela World Wide Web de uma forma metódica e automatizada},
				 plural={\emph{Web Crawlers}}
				}

\newglossaryentry{spider}
				{name={\emph{spider}},
				 description={Programa de computador que navega pela World Wide Web de uma forma metódica e automatizada},
				 plural={\emph{spiders}}
				}
				 
\newglossaryentry{webscraper}
				{name={\emph{Web Scraper}},
				 description={Técnica utilizada para extrair informações de websites. Também chamada de Colheita na Web ou Extração de dados na Web},
				 plural={\emph{Web Scrapers}}
				 }
				 
\newglossaryentry{crawling}
				{name={\emph{crawling}},
				 description={Ato de navegar entre páginas utilizando um \gls{webcrawler}},
				 plural={\emph{web crawling}}
				}

\newglossaryentry{feed}
				{name={\emph{feed}},
				 description={Formato de dados utilizado para fornecer conteúdo atualizado aos usuários.},
				 plural={\emph{feeds}}
				}

\newglossaryentry{CommonLanguageInfrastructure}
				{name={\emph{Common Language Infrastructure}},
				 description={A \emph{Common Language Infrastructure}, ou CLI, é uma especificação aberta desenvolvida pela Microsoft que descreve o código executável e o ambiente de execução que forma o cerne do \emph{framework} Microsoft .NET e de implementações livres e de código aberto como Mono\footnote{\url{http://www.mono-project.com/}} e o Portable.NET \footnote{http://www.gnu.org/software/dotgnu/pnet.html}}.
				}

\newglossaryentry{middleware}
				{name={\emph{middleware}},
				 description={Software ou componente de software que conecta outros componentes ou pessoas a suas aplicações},
				 plural={\emph{middleware}}
				}
				
\newglossaryentry{httpheader}
				{name={header http},
				 description={Campos que contém informações de operação de uma requisição ou resposta HTTP. Definem várias características da transferência de dados}
				}

\newglossaryentry{xpath}
				{name={XPath},
				 description={Linguagem de caminho do XML. É uma linguagem de consulta para seleção de nós em um documento XML. Pode ser usado também para computar valores (como strings, números ou booleanos) no conteúdo de um documento XML. O XPath é definido pelo \emph{World Wide Web Consortium} (W3C).}
				}

\newglossaryentry{debugging}
				{name={\emph{debugging}},
				 description={Processo metódico para encontrar e reduzir o número de bugs ou defeitos em um programa de computador ou em uma peça de \emph{hardware} eletrônico, fazendo assim, que comporte como esperado.}
				}

\newglossaryentry{screenscraping}
				{name={\emph{screen scraping}},
				 description={Técnica na qual um programa de computador extrai dados da saída de vídeo de um outro programa}
				}
				
\newglossaryentry{DomainSpecificLanguage}
				{name={\emph{Domain Specific Language}},
				 description={Linguagem Específica de Domínio - Linguagem de programação ou linguagem de especificação dedicada a um domínio particular de um problema, uma técnica para representação particular de uma problema e/ou uma técnica particular de solução},
				 plural={\emph{Domain Specific Languages}}
				}
				
\newglossaryentry{ReadEvalPrintLoop}
				{name={\emph{Read Eval Print Loop}},
				 description={Ambiente de programação de computadores simples e interativo. O termo é mais utilizado para referenciar o ambiente interativo da linguagem LISP, mas pode ser aplicado para prompts de linha de comando e ambientes similares para as linguagens de programação Smalltalk, Perl, Scala, Python, Ruby, Haskell, Lua, APL, BASIC, J, Tcl e outras linguagens.}
				}

\newglossaryentry{overhead}
				{name={\emph{overhead}},
				 description={Qualquer combinação de uso excessivo de recursos computacionais como tempo de computação, memória, largura de banda ou outros recursos que são necessários para atender um objetivo particular.}
				}

\newglossaryentry{tradeoff}
				{name={\emph{tradeoff}},
				 description={Uma situação que envolve a perda de qualidade ou aspecto de algo em retorno do ganho de outra qualidade ou aspecto. Implica em uma decisão a ser feita com total compreensão de ambos lados bons e ruins de uma escolha em particular.}
				 }