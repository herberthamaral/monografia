% referencial_teorico.tex

\pagebreak
\chapter{Referencial teórico}
\index{Referencial Teórico}

\pagebreak
\section{Recuperação de informação}
\index{Recuperação de informação}

\pagebreak
\section{Web crawlers e web scrapers}
\index{Web crawlers e web scrapers}

\pagebreak
\section{Python}
\index{Python}

\subsection{Histórico}
\index{Python!Historico}
Segundo \cite{pythondoc}, o Python é uma linguagem poderosa, de propósitos gerais, fácil de aprender e programar, interpretada, orientada a objetos e com alguns outros atributos que a tornam uma linguagem ideal para \emph{scripting} e desenvolvimento rápido de aplicações.

A linguagem Python foi criada no início dos anos 1990 por Guido Van Rossum \cite{pythonlicense}, no Stichting Mathematisch Centrum, na Holanda, com o intuito de ser uma linguagem substituta ao ABC, que apresentava uma série de problemas, especialmente com estensibilidade \cite{pythonfaq}. A linguagem Python, inicialmente, foi criada para ser uma substituta da linguagem ABC e com os poderes da API 

Van Rossum permanece como o principal autor da linguagem Python, apesar de receber várias contribuições de colaboradores externos.

Um erro comum é associar a origem o nome da linguagem Python com o nome da serpente. De fato, Van Rossum se inspirou num grupo de comediantes britânicos, o Monty Python, para dar nome à linguagem \cite{pythonfaq} . %referencias?


%colocar mais sobre o histórico

\subsection{Características}
\index{Python!Caracteristicas}
Qualidade de software, produtividade do programador, portabilidade, bibliotecas de suporte, integração de componentes e diversão, são os principais motivos do uso da linguagem \cite{learningpython}. % colocar página 48 e 49

Em seu \emph{design}, o Python implementa uma sintaxe deliberadamente simples e legível e um modelo de programação coerente \cite{learningpython}. %página 50

Dentre as razões históricas, a linguagem Python foi concebida no início da década de 1990, quando ocorreu o \emph{boom} da Internet e quando o número de programadores se tornou escasso para a demanda de Software. Enquanto linguagens de baixo nível como Assembly ou C focam em \emph{produtividade de máquina}, Python foca em \emph{produtividade do programador}. 

Python é deliberadamente otimizada para produtividade do programador. Recursos como sintaxe simples, tipagem dinâmica, falta de necessidade de compilação e um conjunto de ferramentas embutidas permitem que programadores desenvolvam programas em uma fração de tempo necessária em comparação a quando usam outras ferramentas \cite{learningpython}. % página 50

Entretanto, existe um \emph{tradeoff} no uso da linguagem Python: a velocidade de execução. Programadores tendem a ser mais produtivos com o uso de linguagens com os atributos que Python possui. Porém, o código Python não é um código de máquina, portanto o mesmo precisa ser interpretado a cada execução, o que o torna mais lento. 

Outros fatores como a tipagem dinâmica tendem a trazer mais \emph{overhead} na execução dos programas em Python, o que degrada ainda mais sua velocidade de execução, o que torna seu uso difícil ou inviável para projetos que dependam estritamente de velocidade de execução (ex: componentes de baixo nível, como Kernels e drivers de dispositivos, aplicativos de produtividade, como suítes de escritórios e CAD e outros softwares de grande porte).

\subsection{Ambientes e plataformas}
\index{Python!Ambientes e plataformas}

Python é uma linguagem portável e multiplataforma. Isso significa que um código em Python pode ser executado nos mais diversos ambientes e sistemas operacionais, como Windows, Linux, Mac OS e em até sistemas operacionais móveis como Symbian e Android.

Há também a possibilidade de executar a linguagem Python em outros ambientes diferentes do original (CPython). Iniciativas como Jython (Python para a \emph{Java Virtual Machine}) e o IronPython (Python para o ambiente Microsoft.NET) permitem que o Python seja executado dentro das duas das maiores plataformas de desenvolvimento de software da atualidade, aproveitando seus recursos e suas funcionalidades. 

Desta forma, um programa em Python pode utilizar o Swing através do Jython para o desenvolvimento de uma interface gráfica em ambiente Java ou pode utilizar o \emph{Windows Communication Foundation} como \emph{framework} de troca de mensagens no ambiente Microsoft.NET.

\subsubsection{CPython}
\index{Python!Ambientes e plataformas!IronPython}

O CPython é a implementação padrão da linguagem Python e é escrita na linguagem de programação C \cite[p.6]{pypy}. A linguagem é implementada por um compilador que traduz código Python em um código \emph{bytecode} de altíssimo nível (\emph{very high level}) e por uma máquina virtual que interpreta o código.

\subsubsection{IronPython}
\index{Python!Ambientes e plataformas!IronPython}

O IronPython \cite{ironpython} é uma implementação da linguagem de programação Python que é execuatada no framework .NET e Silverlight. Suporta um shell interativo (como a maioria das implementações da linguagem Python) com completa compilação dinâmica. É bem integrado com o resto do framework .NET e torna todas as bibliotecas do .NET facilmente disponíveis para programadores Python, enquanto mantém a compatibilidade com a linguagem Python.


\subsubsection{Jython}
\index{Python!Ambientes e plataformas!Jython}

Jython \cite{jython} é uma implementação da linguagem Python para a JVM (\emph{Java Virtual Machine}). O Jython torna possível a execução da sintaxe da linguagem de programação Python na plataforma Java, o que permite uma integração transparente com as bibliotecas Java e outras aplicações baseadas em Java. 

\subsubsection{PyPy}
\index{Python!Ambientes e plataformas!PyPy}

PyPy é uma implementação do Python escrita em Python \cite[p. 7]{pypy}. A idéia principal é escrever uma especificação de alto nível do interpretador em um subtrato restrito do Python chamado RPython (\emph{Restricted Python}) com o intuito de ser traduzido para executáveis eficientes de baixo nível para o ambiente C/POSIX, JVM e CLI, o que garante a portabilidade. 



\subsection{Python e Computação Científica}
\index{Python!Python e Computação Científica}

Segundo \cite{python_scientific_world}, Python e estensões como o NumPy,Mas Ik, estão se tornando padrão para muitas ciências que precisam processar grande quantidades de dados, desde a neuroimagem à astronomia.

Um dos casos mais famosos da linguagem Python em sistemas de recuperação de informações é o Google.\cite{google}. Python representa um papel fundamental dentro da estrutura de motores de busca do Google, sendo responsável pelos seus \emph{crawlers} e pelos seus servidores de URL \cite{surveyir}.



\pagebreak
\section{Scrapy}
\index{Scrapy}

\pagebreak
\section{Website Parse Template}
\index{Website Parse Template}

\pagebreak
\section{Protocol Buffers}
\index{Protocol Buffers}
