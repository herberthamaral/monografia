% referencial_teorico.tex

\chapter{Referencial teórico}
\index{Referencial Teórico}

\section{Recuperação de informação}
\index{Recuperação de informação}

\section{Web crawlers e web scrapers}
\index{Web crawlers e web scrapers}

\section{Python}
\index{Python}

Segundo \cite{pythondoc}, o Python é uma linguagem poderosa, fácil de aprender e programar, interpretada, orientada a objetos e com alguns outros atributos que a tornam uma linguagem ideal para \emph{scripting} e desenvolvimento rápido de aplicações.

A linguagem Python foi criada no início dos anos 1990 por Guido Van Rossum \cite{pythonlicense}, no Stichting Mathematisch Centrum, na Holanda. Van Rossum permanece como o principal autor da linguagem Python, apesar de receber várias contribuições de colaboradores externos.


\section{Scrapy}
\index{Scrapy}
\section{Website Parse Template}

\section{Protocol Buffers}
\index{Protocol Buffers}
