% resultados.tex
\chapter{Resultados}

O protótipo aqui apresentado ainda encontra-se em desenvolvimento. Por ser uma contribuição ao Scrapy, um framework de código aberto, este protótipo poderá ser desenvolvido por outros membros da comunidade.

Um dos aspectos positivos da abordagem adotada é que a ferramenta desenvolvida gera código na linguagem de programação Python a partir de código em XML, permitindo que o programador faça altarações no código gerado, caso seja necessário. 

Como este trabalho não contempla, tampouco pretende contemplar, todas as facetas da criação de sistemas de recuperação de informações estruturadas na Web, a possibilidade do programador alterar o código gerado pode vir a ser bem útil em situações mais específicas.

A seguir, é mostrado duas avaliações, quantitativas e qualitativas, dos resultados obtidos.

\section{Avaliação quantitativa}

A avaliação de sucesso no cumprimento dos objetivos deste projeto pode ser meramente subjetiva, porém, há alguns critérios quantitativos ou subjetivos que podem ser discutidos, como a economia em linhas de código. No exemplo mais simples possível, há o seguinte template:

\lstset{language=XML,
basicstyle=\scriptsize,
caption={Exemplo de template em WPT},
captionpos=b
}
\begin{lstlisting}
  <ow:wpt xmlns:ow="http://www.omfica.org/schemas/ow/0.9"
            ow:host="http://example.com">
  <ow:template ow:name="Template Example" ow:url="http://www.example.com/index.php">
      <ow:block ow:tagid="ex1" name="ex1"></ow:block>
  </ow:template> 
  </ow:wpt>
\end{lstlisting}

O template anterior gera o seguinte código em Python:

\lstset{language=Python,
basicstyle=\scriptsize,
caption={\emph{Spider} gerado a partir do arquivo de entrada apresentado na listagem 5.3},
captionpos=b
}
\begin{lstlisting}
from scrapy.item import Item, Field

class TemplateExampleItem1(Item):
    ex1 = Field()

from scrapy.spider import BaseSpider
from scrapy.contrib.loader import XPathItemLoader

class TemplateExample(BaseSpider):
    name = 'example.com'
    allowed_domains = ['example.com']
    start_urls = [
        'http://www.example.com/index.php',
    ]
    
    def parse(self, response):
        l = XPathItemLoader(item = TemplateExampleItem1(),response=response)
        l.add_xpath('bubble','id("ex1")/text()') 
        i = l.load_item()
        yield i
\end{lstlisting}

O código em Python possui 20 linhas, 16 sem as linhas em branco, ao passo que o template possui apenas 5 linhas de código. Neste exemplo, na pior das hipóteses, a economia de código gira em torno de 2/3 do original. A diferença fica mais evidente em um exemplo mais completo:

\lstset{language=XML,
basicstyle=\scriptsize,
caption={Exemplo mais completo de template em WPT},
captionpos=b
}
\begin{lstlisting}
<ow:wpt xmlns:ow="http://www.omfica.org/schemas/ow/0.9"
ow:host="http://example.com">
<ow:template ow:name="Template Example" ow:url="http://www.example.com/index.php">
  <ow:block ow:tagid="ex1" name="ex1"></ow:block>
  <ow:block ow:xpath="/html/body/p/text()" name="paragraph"></ow:block>
  
  <ow:block ow:xpath="/html/body/ul/li" name="menu">
    <ow:block ow:xpath="//a/text()" name="item" ow:type="repetable"></ow:block>
    <ow:block ow:xpath="//a[@href]" name="url" ow:type="repetable"></ow:block>
  </ow:block>
  
  <ow:block ow:xpath="/html/body/div[2]/ul/li" name="paginate">
    <ow:block ow:xpath="//a[@href]" name="url" ow:type="follow"></ow:block> <!-- follow -->
  </ow:block>
</ow:template> 
</ow:wpt>
\end{lstlisting}

\pagebreak
\lstset{language=Python,
basicstyle=\scriptsize,
caption={\emph{Spider} gerado a partir do arquivo de entrada apresentado na listagem 5.3},
captionpos=b
}
\begin{lstlisting}
from scrapy.item import Item, Field

class TemplateExampleItem1(Item):
    ex1 = Field()

class TemplateExampleItem2(Item):
    paragraph = Field()
    
class MenuItem(Item):
    item = Field()
    url = Field()
    
class Paginate(Item):
    url = Field()

from scrapy.spider import BaseSpider
from scrapy.contrib.loader import XPathItemLoader
from scrapy.selector import HtmlXPathSelector
from scrapy.http import Request

class TemplateExample(BaseSpider):
    name = 'example.com'
    allowed_domains = ['example.com']
    start_urls = [
        'http://www.example.com/index.php',
    ]

    def parse(self, response):
        l = XPathItemLoader(item = TemplateExampleItem1(),response=response)
        l.add_xpath('bubble','id("ex1")/text()') 
        i = l.load_item()
        yield i
        
        l = XPathItemLoader(item = TemplateExampleItem2(),response=response)
        l.add_xpath('paragraph','/html/body/p/text()') 
        i = l.load_item()
        yield i
        
        hxs = HtmlXPathSelector(response)
        for item in hxs.select('/html/body/ul/li'):
            l = XPathItemLoader(item = MenuItem(), selector=item)
            l.add_xpath('item','//li/a/text()') 
            l.add_xpath('url','//li/a[@href]') 
            i = l.load_item()
            yield i
                        
        hxs = HtmlXPathSelector(response)
        for item in hxs.select('/html/body/div[2]/ul/li'):
            l = XPathItemLoader(item = Paginate(), selector=item)
            l.add_xpath('url','//li/a[@href]') 
            i = l.load_item()
            yield Request(item=i['url'],callback=self.parse)
            yield i
\end{lstlisting}

Neste segundo exemplo, o arquivo XML possui 15 linhas, ao passo que o código gerado possui cerca de 60 linhas, aproximadamente 4 vezes menos código.

\pagebreak
\section{Avaliação qualitativa}

Em uma comparação qualitativa, pode-se citar alguns critérios:

\begin{enumerate}
	\item Linguagens de propósito geral \emph{versus} linguagens de proposíto específico (ou Domain Specific Languages).
	\item Ganho de produtividade.
	\item Facilidade para o usuário final.
\end{enumerate}

Linguagens de propósito geral servem para construção de mais diversos tipos de sistemas, ao passo que linguagens de propósito específico são projetadas para resolver problemas em um domínio específico. Há um \emph{tradeoff} entre as duas abordagens: enquanto as linguagens de propósito geral são mais completas, elas também são mais complexas se comparadas com as linguagens de propósito específico para a mesma categoria de problema.

Quando se deixa de usar uma linguagem de propósito específico (Python) para utilizar outra de propósito geral (XML/WPT) o ganho de produtividade estará em eliminar preocupações não inerentes ao domínio. Com menos recursos e preocupações para se trabalhar, o trabalho do usuário final é facilitado.

O uso de XML pode trazer várias vantagens à criação de sistemas de recuperação de informação. A principal é que qualquer linguagem de programação que dê suporte à XML pode criar templates. Como o trabalho de \emph{não} escrever código é sempre menor do que escrever \emph{algum} código, os usuários podem se beneficiar com a criação de sistemas que geram templates automaticamente. Por exemplo, um \emph{plugin} de um navegador Web pode permitir que o usuário selecione as seções de uma determinada página que gostaria que fossem recuperadas automaticamente e então gerar o template para o mesmo.

\pagebreak
\section{Dificuldades encontradas}

Apesar da abundante documentação e mesmo não possuindo uma má qualidade de código ou baixa cobertura de testes, uma das principais dificuldades foi aderir à arquitetura e estilo de codificação do Scrapy para que a padronização entre componentes fosse mantida.

O framework em si possui um tamanho considerável (759 arquivos em 13/11/2010 \footnote{Informação obtida através do comando \texttt{find | wc -l} }), portanto, leva-se um tempo para acostumar-se com a estrutura de diretórios, arquivos e padrões de nome.

Não foi possível encontrar referências acadêmicas do Scrapy, um dos motivos pelos quais somente a respectiva documentação oficial foi utilizada.

O formato WPT possui apenas informações sobre a disposição dos elementos em uma página e seus respectivos significados (ontologia). Tal fato leva ao incoveniente de haver diferentes \emph{schemas} XML em um mesmo arquivo para determinar outras informações, como nome dos variáveis do código gerado e formato de persistência dos dados obtidos.